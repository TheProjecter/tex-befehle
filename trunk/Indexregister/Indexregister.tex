\section{Indexregister erstellen}
\begin{itemize}
	\item \textcolor{blue}{\tt \textbackslash usepackage\{makeidx\}}
	\item \textcolor{blue}{\tt \textbackslash makeindex}
	\item beides direkt untereinander und noch vor dem Begin des Dokuments!
	\item Einf�gen von \textcolor{red}{\tt \textbackslash printindex} vor dem Dokument-Ende (bzw. an der Stelle, wo das Indexregister erscheinen soll)
	\item An den Stellen/W�rtern, die in den Index eingetragen werden sollen, folgendes anf�gen:\\
			\textcolor{red}{\tt \textbackslash index\{Indexeintrag\_bzw\_Wort\}}
	\item oder \textcolor{red}{\tt \textbackslash index\{Eintrag!Untereintrag\}}
	\item \textcolor{red}{\tt \textbackslash index\{Virtuell@Eintrag\}}\\
			Virtuelle Eintr�ge sind notwendig, um Sonderzeichen oder mathematische Symbole in den Index einzuordnen\\
			\textcolor{orange}{\tt \textbackslash index\{wunschenswert@w�nschenswert\}}\\
			\textcolor{orange}{\tt \textbackslash index\{R@\textbackslash R\}}
\end{itemize}