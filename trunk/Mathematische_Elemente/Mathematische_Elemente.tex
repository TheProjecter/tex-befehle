\newpage
\section{Mathematische Elemente}
\subsection{Wurzeln}
\begin{displaymath}
\begin{array}{lc@{\qquad}l}
&\sqrt{x}  &\backslash sqrt \{ x\}\\
&\sqrt[3]{x}  &\backslash sqrt [3] \{ x\}
\end{array}
\end{displaymath}

\subsection{Br�che und Binomial-Koeffizienten}
\begin{displaymath}
\begin{array}{lc@{\qquad}l}
&\frac{x^2}{y}  &\backslash frac \{ x\;\hat{}\; 2 \} \{ y \}\\
&x^{\frac{1}{2}}  &x\;\hat{}\;\{ \backslash frac \{1\}\{2\}\}\\
&{n \choose k}  &\{ n\;\backslash choose\;k \}\\
&{x \atop y+2}  &\{ x\;\backslash atop\;y+2 \}
\end{array}
\end{displaymath}

\subsection{Waagerechte Striche und Klammern}
\begin{displaymath}
\begin{array}{lc@{\qquad}l}
&\overline{m+n}  &\backslash overline \{ m+n\}\\
&\underbrace{a+b+\cdots+z}_{26}  &\backslash underbrace \{ a+b+\backslash cdots +z\} \_\{26\}
\end{array}
\end{displaymath}

\subsection{Funktionsnamen}
\begin{displaymath}
\begin{array}{lc@{\qquad}l@{\hspace{2.0cm}}c@{\qquad}l@{\hspace{2.0cm}}c@{\qquad}l}
&\arg  	&\backslash arg &\ln  	&\backslash ln	&\sin  		&\backslash sin\\
&\cos  	&\backslash cos &\log  	&\backslash log &\tan  		&\backslash tan\\
&\exp  	&\backslash exp &\max  	&\backslash max &\\
&\lg  	&\backslash lg  &\min  	&\backslash min &
\end{array}
\end{displaymath}

\subsection{Komma als Dezimaltrennzeichen}
Das Komma ist in LaTeX standardm��ig ein Aufz�hlungszeichen. Soll ein Komma als Dezimaltrennzeichen verwendet werden, so kann dies durch geschweifte Klammern bewerkstelligt werden.
\begin{displaymath}
\begin{array}{lc@{\qquad}l}
&3{,}14	&3\{,\}14\\
&3,14	&3,\!14 
\end{array}
\end{displaymath}

\subsection{Matrizen}
Es gibt f�r jede verscheidene Matrizenart eine eigene Umgebung.\\
\begin{LTXexample}[pos=l, rframe={}, width=.5]
$\begin{pmatrix} a & b \\ c & d \end{pmatrix}$
\end{LTXexample}

\begin{LTXexample}[pos=l, rframe={}, width=.5]
$\begin{bmatrix}
 0 & \cdots & 1 \\
 2 & \cdots & 3 
\end{bmatrix}$
\end{LTXexample}

\begin{LTXexample}[pos=l, rframe={}, width=.5]
$\begin{vmatrix}
 a & b \\
 c & d 
\end{vmatrix}$
\end{LTXexample}

\begin{LTXexample}[pos=l, rframe={}, width=.5]
$\begin{Vmatrix}
 0 & 1 \\
 2 & 3 
\end{Vmatrix}$
\end{LTXexample}

\subsection{Mehrzeilige Gleichungen}
\begin{LTXexample}[pos=l, rframe={}, width=.5]
\begin{align*}
V_{Max} & = 12345\\
& = 2345\\
& = 345
\end{align*}
\end{LTXexample}

\begin{LTXexample}[pos=l, rframe={}, width=.5]
\begin{alignat*}{2}
V_{Max} & = 12345 \qquad & \text{Axiom 1}\\
& = 2345 & \text{Axiom 2}\\
& = 345  & \text{Axiom 3}
\end{alignat*}
\end{LTXexample}

\begin{LTXexample}[pos=l, rframe={}, width=.5]
$\begin{array}{lc@{\qquad}l}
1221 & test & 1221\\
23332 & testtext & 23332
\end{array}$
\end{LTXexample}


\subsection{Beispiele}
\begin{displaymath}
\begin{array}{lc@{\qquad}l}
&a \bmod b  &a\;\backslash bmod\;b\\
&x \equiv a \pmod {b}  &x\;\backslash equiv\;a\;\backslash pmod \{b\}\\
&\lim\limits_{x \to 0} \frac{\sin x}{x}  &\backslash lim _{} \{x \backslash to\;0\} \backslash frac \{\backslash sin\;x\}\{x\}\\
&\sum\limits^{\infty}_{n=1}(T \geq n) &\backslash sum\backslash limits\hat\; \{\backslash infty\}\_\{n=1\}(T \backslash geq\;n)
\end{array}
\end{displaymath}