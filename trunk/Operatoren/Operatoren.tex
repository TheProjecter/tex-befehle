\section{Operatoren}
\subsection{Bin�re Operatoren}
\begin{displaymath}
\begin{array}{lc@{\qquad}l@{\hspace{2.0cm}}c@{\qquad}l@{\hspace{2.0cm}}c@{\qquad}l}
&\times  	&\backslash times &\cdot    &\backslash cdot &\circ    &\backslash circ\\
&\mp  	&\backslash mp &\pm  	&\backslash pm &\oplus    &\backslash oplus\\
&\cap  	&\backslash cap &\cup  	&\backslash cup &\odot    &\backslash odot\\
&\vee  	&\backslash vee &\wedge  	&\backslash wedge &\div	&\backslash div
\end{array}
\end{displaymath}

\subsection{"`gro�e"' Operatoren}
\begin{displaymath}
\begin{array}{lc@{\qquad}l@{\hspace{2.0cm}}c@{\qquad}l@{\hspace{2.0cm}}c@{\qquad}l}
&\sum  	&\backslash sum &\prod  	&\backslash prod &\int    &\backslash int\\
&\bigwedge  	&\backslash bigwedge &\bigcup  	&\backslash bigcup
\end{array}
\end{displaymath}

\subsection{Relationen}
\begin{displaymath}
\begin{array}{lc@{\qquad}l@{\hspace{2.0cm}}c@{\qquad}l}
&\leq  	&\backslash leq &\geq  	&\backslash geq\\
&\ll  	&\backslash ll &\gg  	&\backslash gg\\
&\subset    &\backslash subset &\supset    &\backslash supset\\
&\subseteq    &\backslash subseteq &\supseteq    &\backslash supseteq\\
&\in  	&\backslash in &\ni  	&\backslash ni\\
&\notin  	&\backslash notin &\notni  	&\backslash notni\\
&\equiv  	&\backslash equiv &\sim  	&\backslash sim\\
&\approx  	&\backslash approx &\cong  	&\backslash cong\\
&\not=  	&\backslash not= &\not\equiv  	&\backslash not \backslash equiv\\
&\coloneqq  	&\backslash coloneqq &\eqqcolon  	&\backslash eqqcolon\\
&\triangleq  	&\backslash triangleq &\hat{=}	&\backslash hat\{=\}
\end{array}
\end{displaymath}

\subsection{Pfeile}
\begin{displaymath}
\begin{array}{lc@{\qquad}l@{\hspace{2.0cm}}c@{\qquad}l}
&\mapsto  	&\backslash mapsto &\longmapsto  	&\backslash longmapsto\\
&\rightarrow  	&\backslash rightarrow &\longrightarrow  	&\backslash longrightarrow\\
&\Rightarrow  	&\backslash Rightarrow &\longleftarrow  	&\backslash longleftarrow\\
&\leftrightarrow  	&\backslash leftrightarrow &\Longleftrightarrow  	&\backslash Longleftrightarrow\\
&\uparrow  	&\backslash uparrow &\Uparrow  	&\backslash Uparrow\\
&\downarrow  	&\backslash downarrow &\Downarrow  	&\backslash Downarrow\\
&\nearrow  	&\backslash nearrow &\searrow  	&\backslash searrow\\
&\nwarrow  	&\backslash nwarrow &\swarrow  	&\backslash swarrow
\end{array}
\end{displaymath}

\subsection{beschriftete Pfeile}
\begin{LTXexample}[pos=l, rframe={}, width=.45]
\begin{center}
$B \xrightarrow[T]{n\pm i-1} C$
\end{center}
\end{LTXexample}

\subsection{Anordnung �ber-/untereinander}
\begin{LTXexample}[pos=l, rframe={}, width=.45]
\begin{center}
$ \underset{x}{yz} $\\
$ \overset{a}{bcd} $\\
$ \stackrel{!}{=} $
\end{center}
\end{LTXexample}

\subsection{Klammern}
\begin{displaymath}
\begin{array}{lc@{\qquad}l@{\hspace{2.0cm}}c@{\qquad}l}
&\{  &\backslash \{ &\} &\backslash \}\\
&\|  &\backslash \| \\
&\lfloor  &\backslash lfloor &\rfloor &\backslash rfloor\\
&\lceil &\backslash lceil &\rceil &\backslash rceil\\
&\langle  &\backslash langle &\rangle &\backslash rangle\\
&\llfloor  &\backslash llfloor &\rrfloor &\backslash rrfloor
\end{array}
\end{displaymath}
Um mathematische Ausdr�cke mit der richtigen Klammergr��e zu versehen, \quad $\backslash left$ \quad oder \quad $\backslash right$ \quad davor schreiben, z.B.: \quad $\backslash left($ \quad oder \quad$ \textcolor{red}{\backslash right \backslash \}}$.

\subsection{Fallunterscheidungen}
\begin{LTXexample}[pos=l, rframe={}, width=.50]
$f(x) = \begin{cases}
x/2 \qquad & , x \text{ gerade}\\
3x+1 & , x \text{ ungerade}
\end{cases}$
\end{LTXexample}