\newpage
\section{Eigene Kommandos definieren}
\subsection{Allgemein}
Syntax zum definieren von eigenen Kommandos:\\
\textit{$\backslash$newcommand$\{\backslash$neuerName$\}\{$Definition$\}$}\\

Man kann den neuen Kommandos auch Argumente �bergeben (maximal 9 Argumente!):\\
\textit{$\backslash$newcommand$\{\backslash$neuerName$\}$[Anzahl der Argumente]$\{$Definition$\}$}\\


\subsection{Eigene Kommandos}
\begin{LTXexample}[pos=l, rframe={}, width=.45]
\FALSCH \\
\RICHTIG \\
\Aufg{3} \\
\Loes{4} \\
\UAufg{6} \\
\mathee{x^2}
\Mathee{x^2}
x\Igl y \\
x\igl y \\
\longhookrightarrow \\
$\rar$ \\
$\lrr$ \\
vertikale Abst�nde: \\
\ ssp = 2mm \\
\ Ssp = 3mm \\
\ msp = 5mm \\
\ Msp = 8mm \\
\ sspa = -2mm \\
\ Sspa = -3mm \\
\ mspa = -5mm \\
\ Mspa = -8mm \\
\einruck{einger�ckter Text}
\txt{f�r Text in Math-Umgebungen} \\
\tdgrey{farbiger Text} \\
\tblue{auch in red, orange und dgreen} \\
$\vek{0 & 1\\ 2 & 3}$ \\
\cen{zentrierter Text}
$\bck = \backslash $ \\
$\mbbn  \mbbr  \mbbz  \mbbc  \mbbq  \mbbp$ \\
$\defigl$ \\
$x \defmit x$
\end{LTXexample}
