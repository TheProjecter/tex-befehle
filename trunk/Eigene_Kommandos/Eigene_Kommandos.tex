\newpage
\section{Eigene Kommandos definieren}
\subsection{Allgemein}
Syntax zum definieren von eigenen Kommandos:\\
\textit{$\backslash$newcommand$\{\backslash$neuerName$\}\{$Definition$\}$}\\

Man kann den neuen Kommandos auch Argumente �bergeben (maximal 9 Argumente!):\\
\textit{$\backslash$newcommand$\{\backslash$neuerName$\}$[Anzahl der Argumente]$\{$Definition$\}$}\\


\subsection{Eigene Kommandos}
\begin{LTXexample}[pos=l, rframe={}, width=.45]
\FALSCH \\
\RICHTIG \\
\Aufg{3} \\
\Loes{4} \\
\UAufg{6} \\
\mathee{x^2} \\
\igl \\
\txt{f�r Text in Math-Umgebungen} \\
\end{LTXexample}
