\section{Abst�nde}
\subsection{Horizontale Abst�nde}
\begin{displaymath}
\begin{array}{ll@{\qquad\quad}l}
&\backslash!  &xx|\!|xx\\
&\textrm{normal ohne LZ} &xx||xx\\
&\backslash,  &xx|\,|xx\\
&\backslash:  &xx|\:|xx\\
&\textrm{normal mit LZ} &xx|\textrm{ }|xx\\
&\backslash;  &xx|\;|xx\qquad\textrm{(Leerzeichen)}\\
&\backslash enspace  &xx|\enspace|xx\\
&\backslash quad  &xx|\quad|xx\\
&\backslash qquad  &xx|\qquad|xx\\
&\backslash hspace\{1.0cm\}  &xx|\hspace{1.0cm}|xx
\end{array}
\end{displaymath}

\subsection{Vertikale Abst�nde}
\begin{displaymath}
\begin{array}{ll@{\qquad}l}
&\backslash smallskip  	&\textrm{etwa 1/4 Zeile}\\
&\backslash medskip 	&\textrm{etwa 1/2 Zeile}\\
&\backslash bigskip 	&\textrm{etwa 1 Zeile}\\
&\backslash vfill 	&\textrm{Abstand zwischen 0 und unendlich}\\
&\backslash vspace\{n\} 	&\textrm{Ein n hoher Abstand} \\
&\backslash vspace^*\{n\} 	&\textrm{\textbf{Erzwingen} eines n hohen Abstandes} \\
&\backslash addvspace\{\}	 &\textrm{zus�tzlicher Abstand zwischen Abs�tzen}\\
&\backslash newpage 	&\textrm{Seitenwechsel} \\
&\backslash\backslash 	&\textrm{Zeilenwechsel} \\
& Leerzeile   &\textrm{neuer Absatz (ggf. mit Einr�ckung) und Zeilenwechsel}
\end{array}
\end{displaymath}


\subsection{Seiteumbr�che}
\begin{description}
\item[\textcolor{blue}{\tt \bck pagebreak}] empfiehlt einen Seitenumbruch an angegebenen Stelle. Die Abst�nde auf der Seite werden so gesetzt, dass die Seite b�ndig mit Kopf und Fu� abschlie�t.
\item[\textcolor{blue}{\tt \bck newpage}] erzwingt einen Seitenumbruch an angegebenen Stelle. Die Abst�nde auf der Seite werden nicht vergr��ert, um fu�b�ndig zu werden, sondern es wird ggf. nach unten Platz gelassen.
\item[\textcolor{blue}{\tt \bck clearpage}]  wie {\tt \bck newpage}, aber erzwingt zus�tzlich die Ausgabe aller noch nicht gesetzten \textit{Gleitobjekte} (Abbildungen, Tabellen) auf den n�chsten Seiten.
\item[\textcolor{blue}{\tt \bck cleardoublepage}]  wie {\tt \bck clearpage}, aber f�gt notfalls noch eine weitere Leerseite ein, damit der n�chste Text auf einer Seite mit ungerader Seitenzahl beginnt (sinnvoll nur f�r doppelseitigen Druck).
\end{description}


