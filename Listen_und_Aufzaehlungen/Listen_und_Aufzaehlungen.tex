\newpage
\section{Listen und Aufz�hlungen}
\subsection{Aufz�hlung mit Punkten}
\begin{LTXexample}[pos=l, rframe={}, width=.45]
\begin{itemize}
	\item Punkt 1
	\begin{itemize}
		\item Punkt 1
		\item Punkt 2
		\begin{itemize}
			\item Punkt 1
			\item Punkt 2
		\end{itemize}
	\end{itemize}
	\item Punkt 2
	\item blablabla
\end{itemize}
\end{LTXexample}


\subsection{Aufz�hlung mit Nummern}
\begin{LTXexample}[pos=l, rframe={}, width=.45]
\begin{enumerate}
	\item Punkt 1
	\begin{enumerate}
		\item Punkt 1
		\item Punkt 2
		\begin{enumerate}
			\item Punkt 1
			\item Punkt 2
		\end{enumerate}
	\end{enumerate}
	\item Punkt 2
	\item blablabla
\end{enumerate}
\end{LTXexample}



\subsection{Aufz�hlung mit Markierungsworten}
ist fast sowas wie ein Stickwortverzeichnis:
\begin{LTXexample}[pos=l, rframe={}, width=.45]
\begin{description}
	\item[Eselsturm,] der: an romanischen Kirchen mit gewendelter
		Rampe, auf der das Baumaterial von Eseln hinaufgetragen wurde.
	\item[Feld,] das: viereckige, polygonale oder krummlinig
		umrahmte Fl�che an W�nden, Decken, Gew�lben.
	\item[Grede,] die (lat.): Freitreppe.
\end{description}
\end{LTXexample}



\subsection{Aufz�hlung mit benutzerdefinierten Zeichen}
Diese Zeichen ben�tigen das Package "`bbding"'.\\
\begin{LTXexample}[pos=l, rframe={}, width=.45]
\begin{itemize}
	\item[\HandRight] Punkt 1
	\item[\PencilRight] Punkt 2
	\item[\XSolidBrush] Punkt 3
	\item[\DiamondSolid] Punkt 4
	\item[\OrnamentDiamondSolid] Punkt 5
	\item[\ArrowBoldRightStrobe] Punkt 6
	\item[\ArrowBoldDownRight] Punkt 7
	\item[\CircleSolid] Punkt 8
	\item[\Square] Punkt 9
	\item[\SquareSolid] Punkt 10
\end{itemize}
\end{LTXexample}



\subsection{Ver�nderung der Standard-Zeichen}
\begin{LTXexample}[pos=l, rframe={}, width=.45]
\renewcommand{\labelitemi}{$\circ$}
\renewcommand{\labelitemii}{$\bullet$}
\renewcommand{\labelitemiii}{$\diamond$}
\begin{itemize}
	\item Punkt xx
	\begin{itemize}
		\item Punkt yy
		\item Punkt yyy
		\begin{itemize}
			\item Punkt 1
			\item Punkt 2
		\end{itemize}
	\end{itemize}
	\item Punkt 2
	\item blablabla
\end{itemize}
\end{LTXexample}

\begin{LTXexample}[pos=l, rframe={}, width=.45]
\renewcommand{\labelenumi}{\alph{enumi})}
\renewcommand{\labelenumii}{\alph{enumi}.\Roman{enumii}}
\renewcommand{\labelenumiii}{\arabic{enumiii}.)}
\begin{enumerate}
	\item Punkt 1
	\begin{enumerate}
		\item Punkt 1
		\item Punkt 2
		\begin{enumerate}
			\item Punkt 1
			\item Punkt 2
		\end{enumerate}
	\end{enumerate}
	\item Punkt 2
	\item blablabla
\end{enumerate}
\end{LTXexample}
