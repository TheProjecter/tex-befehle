\documentclass[a4paper, 10pt, DIV15, parskip=half-]{scrartcl}  % halfparskip = halbzeiliger Absatzabstand

\listfiles

\usepackage[ngerman]{babel}

\usepackage[T1]{fontenc}

\usepackage[latin1]{inputenc}

%++++++++++++++++++++++++++++++++++++++++++++++++++++++++++++++++++
%+++ Schriften ++++++++++++++++++++++++++++++++++++++++++++++++++++
%++++++++++++++++++++++++++++++++++++++++++++++++++++++++++++++++++
\usepackage{charter}
\usepackage[scaled=.92]{helvet}
\usepackage{courier}

\usepackage{microtype}


%++++++++++++++++++++++++++++++++++++++++++++++++++++++++++++++++++
%+++ weitere Pakete +++++++++++++++++++++++++++++++++++++++++++++++
%++++++++++++++++++++++++++++++++++++++++++++++++++++++++++++++++++
 % Paket um Bilder einzubinden. Dokumentation: l2picfaq.pdf
\usepackage{graphicx}

% mehrspaltiger Textsatz - allerdings nicht im ganzen Dokument
\usepackage{multicol}

% zum Anzeigen von Internetlinks
\usepackage{url}

% einige spezielle Mathe-Kommandos
\usepackage{amsmath}
\usepackage{amsfonts}
\usepackage{stmaryrd}
%\usepackage{amssymb}
%\usepackage{latexsym}

% anzeigen von Quellcode
\usepackage{listings}

% um Formeln (und Text) vom amsmath-package teilweise farbig machen zu k�nnen
\usepackage[svgnames]{xcolor} 
\usepackage{color}

% um gleichzeitig auf einfache Art und Weise den Latex-COde darstellen zu k�nnen
\usepackage[final]{showexpl}

% Definition f�r die Latex-Code-Boxen
\lstset{% 
		backgroundcolor=\color{lightgray},
		breaklines=true
}%

% diverse Symbole z.B. f�r Listen und Aufz�hlungen
\usepackage{bbding}

\usepackage{txfonts} 				% f�r die mathematischen Symbole f�r 'Nicht-Element-von'
\usepackage{ulsy}						% f�r Widerspruchsblitze

% um Tabulatoren in Listen verwenden zu k�nnen
\usepackage{listliketab}

% um Tabellen fester Gesamtbreite definieren zu k�nnen
\usepackage{tabularx}

% f�r mehrseitige Tabellen
\usepackage{longtable}

% f�r farbige Boxen und Rahmen
\usepackage{framed}

% um Graphen zu "zeichnen"
\usepackage{tikz}
\usetikzlibrary{automata,positioning,decorations,shadows,fadings,arrows,snakes,backgrounds,petri,shapes.geometric}


\usepackage{C://Users/Cheffe/Documents/myown}


% Paragraph-�berschirften auch nummerieren
\setcounter{secnumdepth}{5}
% Paragraph-�berschriften auch ins Inhaltsverzeichnis mit aufnehmen
\setcounter{tocdepth}{5}


%++++++++++++++++++++++++++++++++++++++++++++++++++++++++++++++++++
%+++ Kopf- und Fusszeilen +++++++++++++++++++++++++++++++++++++++++
%++++++++++++++++++++++++++++++++++++++++++++++++++++++++++++++++++
\usepackage{scrpage2}

% Hier folgen die Kopfzeilentexte
\ihead{Christian Schwabe}
\ohead{LMU M�nchen}
\ifoot{} %  = Kolumnentitel
\ofoot{\pagemark} %  = Seitenzahl
\cfoot{}


\setheadsepline{0.5pt} % Dicke der Trennlinie Kopfzeile - Text
\setfootsepline{0.5pt} % Dicke der Trennlinie Fusszeile - Text

\pagestyle{scrheadings}	% gemachte Einstellungen anwenden



%++++++++++++++++++++++++++++++++++++++++++++++++++++++++++++++++++
%+++ Zeilenabstand ++++++++++++++++++++++++++++++++++++++++++++++++
%++++++++++++++++++++++++++++++++++++++++++++++++++++++++++++++++++
\usepackage{setspace}
\singlespacing	% 1.5-facher Zeilenabstand; 1 = \singlespacing; 2 = \doublespacing

% H�he der Kopfzeile
\setlength{\headheight}{4em}




% ganz zum Schluss noch das Paket f�r Links und Verlinkungen
\usepackage{hyperref}
\hypersetup{
pdftitle={TeX Befehle},
pdfauthor={Christian Schwabe},
pdfpagelayout=SinglePage,
breaklinks=true,
colorlinks=true,
linkcolor=black,
anchorcolor=rblack,
citecolor=black,
filecolor=black,
menucolor=black,
pagecolor=black,
urlcolor=black 
}
%++++++++++++++++++++++++++++++++++++++++++++++++++++++++++++++++++
%++++++++++++++++++++++++++++++++++++++++++++++++++++++++++++++++++
%+++ hier beginnt das eigenliche Dokument +++++++++++++++++++++++++
%++++++++++++++++++++++++++++++++++++++++++++++++++++++++++++++++++
%++++++++++++++++++++++++++++++++++++++++++++++++++++++++++++++++++
\begin{document}


%++++++++++++++++++++++++++++++++++++++++++++++++++++++++++++++++++
%+++ Titelseite +++++++++++++++++++++++++++++++++++++++++++++++++++
%++++++++++++++++++++++++++++++++++++++++++++++++++++++++++++++++++

\begin{titlepage}
\begin{center}
\vspace*{\fill}{LMU M�nchen\\ Informatik}
\vfill {{\Large\LaTeX{} "~ Befehle \\ \large Eine kleine Auswahl}}
\vfill {Christian Schwabe \\ christian.schwabe@campus.lmu.de \\ \today}
\end{center}
\end{titlepage}



%++++++++++++++++++++++++++++++++++++++++++++++++++++++++++++++++++
%+++ Inhaltsverzeichnis +++++++++++++++++++++++++++++++++++++++++++
%++++++++++++++++++++++++++++++++++++++++++++++++++++++++++++++++++
% Der folgende Befehl generiert automatisch ein Inhaltsverzeichnis, wobei er die
% mittels \section{}... gemachten Titel verwendet. Damit das Inhaltsverzeichnis
% korrekt erstellt wird, muss mindestens 2 mal neu kompiliert werden!
\tableofcontents

% den Rest der Seite frei lassen: \newpage erzeugt lediglich eine neue Seite,
% \clearpage f�gt zus�tzlich noch alle figures, tables ... vorher ein.
\clearpage


%++++++++++++++++++++++++++++++++++++++++++++++++++++++++++++++++++
%+++ Text +++++++++++++++++++++++++++++++++++++++++++++++++++++++++
%++++++++++++++++++++++++++++++++++++++++++++++++++++++++++++++++++
% Eine �berschrift erster Ordnung machen
\section{Installation}
\subsection{MikTeX + TeXnicCenter}
In den Umgebungsvariablen (System -> Erweiterte Systemeinstellungen -> Erweitert -> Umgebungsvariablen -> Path)\\
den Pfad zum MikTeX-bin-Ordner hinzuf�gen: \textcolor{red}{\texttt{C:$\backslash$Program Files$\backslash$MikTeX$\backslash$miktex$\backslash$bin$\backslash$;}}

\subsection{Verwendung von TikZ und TeXnicCenter}
\begin{enumerate}
	\item \textcolor{red}{gnuplot} downloaden und ins C-Programmverzeichnis entpacken: \href{http://www.gnuplot.info/}{\textcolor{blue}{http://www.gnuplot.info/}} 
	\item In den Umgebungsvariablen (System -> Erweiterte Systemeinstellungen -> Erweitert -> Umgebungsvariablen -> Path)\\
den Pfad zum gnuplot-binary-Ordner hinzuf�gen: \textcolor{red}{\texttt{C:$\backslash$Program Files$\backslash$gnuplot$\backslash$binary$\backslash$}}
	\item Im Programm TeXnicCenter eine Kommandozeilen Option hinzuf�gen:\\
			\texttt{-interaction=nonstopmode} $\enspace\longrightarrow\enspace$ \texttt{--src -interaction=nonstopmode --enable-write18 "`\%Wm"'}\\
			unter Ausgabe -> Augabeprofile definieren -> Argumente die an den Compiler �bergeben werden sollen\\
\end{enumerate}

\subsection{Verwendung von PDF-XChange Viewer und TeXnicCenter}
Bei Verwendung des PDF-XChange Viewer als Standard-PDF-Viewer in TeXnicCenter, k�nnen die PDF-Dokumente vor dem Kompilieren automatisch geschlossen werden:
\begin{enumerate}
	\item In TexnicCenter auf \textbf{Ausgabe} klicken
	\item \textbf{Ausgabeprofil definieren} 
	\item \textbf{Viewer}
	\item Beim \textbf{Pfad der Anwendung} den Pfad zum Viewer angeben, z.B: \textcolor{red}{\texttt{D:$\backslash$Programme$\backslash$Tracker Software$\backslash$PDF-XChange Viewer$\backslash$pdf-viewer$\backslash$PDFXCview.exe}}
	\item \textbf{Projektausgabe betrachten} $\longrightarrow$ Kommandozeile ausw�hlen und als Kommando \textcolor{red}{\texttt{"'\%bm.pdf"'}} eintragen 
	\item Bei \textbf{Compilierung vor Ausgabe schlie�en} auch Kommandozeile ausw�hlen und \textcolor{red}{\texttt{/close "'\%bm.pdf"'}} eintragen\\
\end{enumerate}

\subsection{Verwendung von SumatraPDF und TeXnicCenter (incl. SyncTeX)}
Sumatra PDF ist ein extrem schlanker PDF-Viewer, der �nderungen an ge�ffneten PDF akzeptiert und sogar Vorw�rts- und R�ckw�rtssuche untest�tzt, d.h. das Springen zwischen entsprechenden Stellen in der PDF-Ausgabe und dem Quellcode.\\
Nach dem Installieren von Sumatra PDF mit den Standardeinstellungen wird das Profil LaTeX $\Rightarrow$ PDF mit dem neuen Namen \textcolor{red}{LaTeX $\Rightarrow$ PDF (Sumatra)} kopiert. Folgende Einstellungen werden ge�ndert:
\begin{itemize}
	\item \textbf{(La)Tex}\\Bei \textbf{Argumente, die an den Compiler �bergeben werden sollen} muss folgendes eingetragen werden:\\\textcolor{red}{\texttt{-synctex=-1 -interaction=nonstopmode "'\%pm"'}}
	\item \textbf{Viewer}\\Bei \textbf{Pfad der Anwendung} muss folgendes stehen:\\
		\textcolor{red}{\texttt{C:$\backslash$Programme$\backslash$SumatraPDF$\backslash$SumatraPDF.exe -reuse-instance -inverse-search}}\\		\textcolor{red}{\texttt{"$\backslash$"C:$\backslash$Programme$\backslash$TeXnicCenter$\backslash$TEXCNTR.EXE$\backslash$" /ddecmd $\backslash$"[goto('\%f', '\%l')]$\backslash$"'"'}}\\Hinweis: nach dem \texttt{search} kommt ein Leerzeichen!
	\item Bei \textbf{Projektausgabe betrachten} muss der Radiobutton bei \textcolor{red}{Kommandozeile} gesetzt werden.\\
			Au�erdem ist das Kommando \textcolor{red}{\texttt{[Open("'\%bm.pdf"',0,0,1)]}} einzugeben.
	\item Bei \textbf{Suche in Ausgabe} muss der Radiobutton bei \textcolor{red}{DDE-Kommando} gesetzt werden mit:\\ \textbf{Server:} 	\textcolor{red}{\texttt{SUMATRA}}\\\textbf{Thema:} \textcolor{red}{\texttt{Control}}\\
			Au�erdem ist das Kommando \textcolor{red}{\texttt{[ForwardSearch("'\%bm.pdf"',"'\%Wc"',\%l,0,0,0)]}} einzugeben.
	\item Bei \textbf{Vor Kompilierung Ausgabe schlie�en} muss der Radiobutton bei \textcolor{red}{Nicht schlie�en} gesetzt werden.\\
\end{itemize}

\section{Abst�nde}
\subsection{Horizontale Abst�nde}
\begin{displaymath}
\begin{array}{ll@{\qquad\quad}l}
&\backslash!  &xx|\!|xx\\
&\textrm{normal ohne LZ} &xx||xx\\
&\backslash,  &xx|\,|xx\\
&\backslash:  &xx|\:|xx\\
&\textrm{normal mit LZ} &xx|\textrm{ }|xx\\
&\backslash;  &xx|\;|xx\qquad\textrm{(Leerzeichen)}\\
&\backslash enspace  &xx|\enspace|xx\\
&\backslash quad  &xx|\quad|xx\\
&\backslash qquad  &xx|\qquad|xx\\
&\backslash hspace\{1.0cm\}  &xx|\hspace{1.0cm}|xx
\end{array}
\end{displaymath}

\subsection{Vertikale Abst�nde}
\begin{displaymath}
\begin{array}{ll@{\qquad}l}
&\backslash smallskip  	&\textrm{etwa 1/4 Zeile}\\
&\backslash medskip 	&\textrm{etwa 1/2 Zeile}\\
&\backslash bigskip 	&\textrm{etwa 1 Zeile}\\
&\backslash vfill 	&\textrm{Abstand zwischen 0 und unendlich}\\
&\backslash vspace\{n\} 	&\textrm{Ein n hoher Abstand} \\
&\backslash vspace^*\{n\} 	&\textrm{\textbf{Erzwingen} eines n hohen Abstandes} \\
&\backslash addvspace\{\}	 &\textrm{zus�tzlicher Abstand zwischen Abs�tzen}\\
&\backslash newpage 	&\textrm{Seitenwechsel} \\
&\backslash\backslash 	&\textrm{Zeilenwechsel} \\
& Leerzeile   &\textrm{neuer Absatz (ggf. mit Einr�ckung) und Zeilenwechsel}
\end{array}
\end{displaymath}


\subsection{Seiteumbr�che}
\begin{description}
\item[\textcolor{blue}{\tt \bck pagebreak}] empfiehlt einen Seitenumbruch an angegebenen Stelle. Die Abst�nde auf der Seite werden so gesetzt, dass die Seite b�ndig mit Kopf und Fu� abschlie�t.
\item[\textcolor{blue}{\tt \bck newpage}] erzwingt einen Seitenumbruch an angegebenen Stelle. Die Abst�nde auf der Seite werden nicht vergr��ert, um fu�b�ndig zu werden, sondern es wird ggf. nach unten Platz gelassen.
\item[\textcolor{blue}{\tt \bck clearpage}]  wie {\tt \bck newpage}, aber erzwingt zus�tzlich die Ausgabe aller noch nicht gesetzten \textit{Gleitobjekte} (Abbildungen, Tabellen) auf den n�chsten Seiten.
\item[\textcolor{blue}{\tt \bck cleardoublepage}]  wie {\tt \bck clearpage}, aber f�gt notfalls noch eine weitere Leerseite ein, damit der n�chste Text auf einer Seite mit ungerader Seitenzahl beginnt (sinnvoll nur f�r doppelseitigen Druck).
\end{description}




\section{Mathematische Symbole}
\subsection{Indizes und Potenzen}
\begin{displaymath}
\begin{array}{lc@{\qquad}l}
&x^2	&x\hat\; 2\\
&y_{n+1}	&y\_ \{ n+1 \}
\end{array}
\end{displaymath}


\subsection{Mathematische Akzente}
\begin{displaymath}
\begin{array}{lc@{\qquad}l}
&\vec a   &\backslash vec\:a\\
&\dot a   &\backslash dot\:a\\
&\ddot a   &\backslash ddot\:a\\
&\overline{coin}   &\backslash bar\{coin\}
\end{array}
\end{displaymath}

\subsection{kleine griechische Buchstaben}
\begin{displaymath}
\begin{array}{lc@{\qquad}l@{\hspace{2.0cm}}c@{\qquad}l@{\hspace{2.0cm}}c@{\qquad}l}
&\alpha  	&\backslash alpha &\lambda  	&\backslash lambda &\phi  	&\backslash phi\\
&\beta  	&\backslash beta &\mu  	&\backslash mu &\varphi  	&\backslash varphi\\
&\gamma  	&\backslash gamma &\xi  	&\backslash xi &\omega  	&\backslash omega\\
&\delta  	&\backslash delta &\pi  	&\backslash pi\\
&\epsilon  	&\backslash epsilon &\sigma  	&\backslash sigma\\
&\varepsilon  	&\backslash varepsilon &\tau  	&\backslash tau
\end{array}
\end{displaymath}

\subsection{gro�e griechische Buchstaben}
\begin{displaymath}
\begin{array}{lc@{\qquad}l@{\hspace{2.0cm}}c@{\qquad}l@{\hspace{2.0cm}}c@{\qquad}l}
&\Delta  	&\backslash Delta &\Phi  	&\backslash Phi &\Omega  	&\backslash Omega\\
&\Theta  	&\backslash Theta &\Psi  	&\backslash Psi &\Gamma  	&\backslash Gamma
\end{array}
\end{displaymath}

\subsection{Zahlenbereiche}
\begin{displaymath}
\begin{array}{lc@{\qquad}l@{\hspace{2.0cm}}c@{\qquad}l@{\hspace{2.0cm}}c@{\qquad}l}
&\mathbb{C}  	&\backslash mathbb\{ C \} &\mathbb{Q}  	&\backslash mathbb\{ Q \} &\mathbb{Z}  	&\backslash mathbb\{ Z \}\\
&\mathbb{N}  	&\backslash mathbb\{ N \} &\mathbb{P}  	&\backslash mathbb\{ P \}
\end{array}
\end{displaymath}

\subsection{sonstige Symbole}
\begin{displaymath}
\begin{array}{lc@{\qquad}l@{\hspace{2.0cm}}c@{\qquad}l}
&\imath  	&\backslash imath &\emptyset  	&\backslash emptyset\\
&\jmath  	&\backslash jmath &\nabla  	&\backslash nabla\\
&\wp  	&\backslash wp &\triangle  	&\backslash triangle\\
&\Re  	&\backslash Re \\
&\Im  	&\backslash Im &\forall  	&\backslash forall\\
&\partial  	&\backslash partial &\exists  	&\backslash exists\\
&\infty  	&\backslash infty &\neg    &\backslash neg\\
&\checkmark  	&\backslash checkmark &\Join  	&\backslash Join\\
&\surd  	&\backslash surd &\Box  	&\backslash Box\\
&\parallel  	&\backslash parallel &\interleave  	&\backslash interleave\\
&\ast  	&\backslash ast &\prime  	&\backslash prime
\end{array}
\end{displaymath}

\section{Operatoren}
\subsection{Bin�re Operatoren}
\begin{displaymath}
\begin{array}{lc@{\qquad}l@{\hspace{2.0cm}}c@{\qquad}l@{\hspace{2.0cm}}c@{\qquad}l}
&\times  	&\backslash times &\cdot    &\backslash cdot &\circ    &\backslash circ\\
&\mp  	&\backslash mp &\pm  	&\backslash pm &\oplus    &\backslash oplus\\
&\cap  	&\backslash cap &\cup  	&\backslash cup &\odot    &\backslash odot\\
&\vee  	&\backslash vee &\wedge  	&\backslash wedge &\div	&\backslash div
\end{array}
\end{displaymath}

\subsection{"`gro�e"' Operatoren}
\begin{displaymath}
\begin{array}{lc@{\qquad}l@{\hspace{2.0cm}}c@{\qquad}l@{\hspace{2.0cm}}c@{\qquad}l}
&\sum  	&\backslash sum &\prod  	&\backslash prod &\int    &\backslash int\\
&\bigwedge  	&\backslash bigwedge &\bigcup  	&\backslash bigcup
\end{array}
\end{displaymath}

\subsection{Relationen}
\begin{displaymath}
\begin{array}{lc@{\qquad}l@{\hspace{2.0cm}}c@{\qquad}l}
&\leq  	&\backslash leq &\geq  	&\backslash geq\\
&\ll  	&\backslash ll &\gg  	&\backslash gg\\
&\subset    &\backslash subset &\supset    &\backslash supset\\
&\subseteq    &\backslash subseteq &\supseteq    &\backslash supseteq\\
&\in  	&\backslash in &\ni  	&\backslash ni\\
&\notin  	&\backslash notin &\notni  	&\backslash notni\\
&\equiv  	&\backslash equiv &\sim  	&\backslash sim\\
&\approx  	&\backslash approx &\cong  	&\backslash cong\\
&\not=  	&\backslash not= &\not\equiv  	&\backslash not \backslash equiv\\
&\coloneqq  	&\backslash coloneqq &\eqqcolon  	&\backslash eqqcolon\\
&\triangleq  	&\backslash triangleq &\hat{=}	&\backslash hat\{=\}
\end{array}
\end{displaymath}

\subsection{Pfeile}
\begin{displaymath}
\begin{array}{lc@{\qquad}l@{\hspace{2.0cm}}c@{\qquad}l}
&\mapsto  	&\backslash mapsto &\longmapsto  	&\backslash longmapsto\\
&\rightarrow  	&\backslash rightarrow &\longrightarrow  	&\backslash longrightarrow\\
&\Rightarrow  	&\backslash Rightarrow &\longleftarrow  	&\backslash longleftarrow\\
&\leftrightarrow  	&\backslash leftrightarrow &\Longleftrightarrow  	&\backslash Longleftrightarrow\\
&\uparrow  	&\backslash uparrow &\Uparrow  	&\backslash Uparrow\\
&\downarrow  	&\backslash downarrow &\Downarrow  	&\backslash Downarrow\\
&\nearrow  	&\backslash nearrow &\searrow  	&\backslash searrow\\
&\nwarrow  	&\backslash nwarrow &\swarrow  	&\backslash swarrow\\
&\hookrightarrow  	&\backslash hookrightarrow &\longhookrightarrow  	&\backslash longhookrightarrow
\end{array}
\end{displaymath}

\subsection{beschriftete Pfeile}
\begin{LTXexample}[pos=l, rframe={}, width=.45]
\begin{center}
$B \xrightarrow[T]{n\pm i-1} C$
\end{center}
\end{LTXexample}

\subsection{Anordnung �ber-/untereinander}
\begin{LTXexample}[pos=l, rframe={}, width=.45]
\begin{center}
$ \underset{x}{yz} $\\
$ \overset{a}{bcd} $\\
$ \stackrel{!}{=} $
\end{center}
\end{LTXexample}

\subsection{Klammern}
\begin{displaymath}
\begin{array}{lc@{\qquad}l@{\hspace{2.0cm}}c@{\qquad}l}
&\{  &\backslash \{ &\} &\backslash \}\\
&\|  &\backslash \| \\
&\lfloor  &\backslash lfloor &\rfloor &\backslash rfloor\\
&\lceil &\backslash lceil &\rceil &\backslash rceil\\
&\langle  &\backslash langle &\rangle &\backslash rangle\\
&\llfloor  &\backslash llfloor &\rrfloor &\backslash rrfloor
\end{array}
\end{displaymath}
Um mathematische Ausdr�cke mit der richtigen Klammergr��e zu versehen, \quad $\backslash left$ \quad oder \quad $\backslash right$ \quad davor schreiben, z.B.: \quad $\backslash left($ \quad oder \quad$ \textcolor{red}{\backslash right \backslash \}}$.

\subsection{Fallunterscheidungen}
\begin{LTXexample}[pos=l, rframe={}, width=.50]
$f(x) = \begin{cases}
x/2 \qquad & , x \text{ gerade}\\
3x+1 & , x \text{ ungerade}
\end{cases}$
\end{LTXexample}

\newpage
\section{Mathematische Elemente}
\subsection{Wurzeln}
\begin{displaymath}
\begin{array}{lc@{\qquad}l}
&\sqrt{x}  &\backslash sqrt \{ x\}\\
&\sqrt[3]{x}  &\backslash sqrt [3] \{ x\}
\end{array}
\end{displaymath}

\subsection{Br�che und Binomial-Koeffizienten}
\begin{displaymath}
\begin{array}{lc@{\qquad}l}
&\frac{x^2}{y}  &\backslash frac \{ x\;\hat{}\; 2 \} \{ y \}\\
&x^{\frac{1}{2}}  &x\;\hat{}\;\{ \backslash frac \{1\}\{2\}\}\\
&{n \choose k}  &\{ n\;\backslash choose\;k \}\\
&{x \atop y+2}  &\{ x\;\backslash atop\;y+2 \}
\end{array}
\end{displaymath}

\subsection{Waagerechte Striche und Klammern}
\begin{displaymath}
\begin{array}{lc@{\qquad}l}
&\overline{m+n}  &\backslash overline \{ m+n\}\\
&\underbrace{a+b+\cdots+z}_{26}  &\backslash underbrace \{ a+b+\backslash cdots +z\} \_\{26\}
\end{array}
\end{displaymath}

\subsection{Funktionsnamen}
\begin{displaymath}
\begin{array}{lc@{\qquad}l@{\hspace{2.0cm}}c@{\qquad}l@{\hspace{2.0cm}}c@{\qquad}l}
&\arg  	&\backslash arg &\ln  	&\backslash ln	&\sin  		&\backslash sin\\
&\cos  	&\backslash cos &\log  	&\backslash log &\tan  		&\backslash tan\\
&\exp  	&\backslash exp &\max  	&\backslash max &\\
&\lg  	&\backslash lg  &\min  	&\backslash min &
\end{array}
\end{displaymath}

\subsection{Komma als Dezimaltrennzeichen}
Das Komma ist in LaTeX standardm��ig ein Aufz�hlungszeichen. Soll ein Komma als Dezimaltrennzeichen verwendet werden, so kann dies durch geschweifte Klammern bewerkstelligt werden.
\begin{displaymath}
\begin{array}{lc@{\qquad}l}
&3{,}14	&3\{,\}14\\
&3,14	&3,\!14 
\end{array}
\end{displaymath}

\subsection{Matrizen}
Es gibt f�r jede verscheidene Matrizenart eine eigene Umgebung.\\
\begin{LTXexample}[pos=l, rframe={}, width=.5]
$\begin{pmatrix} a & b \\ c & d \end{pmatrix}$
\end{LTXexample}

\begin{LTXexample}[pos=l, rframe={}, width=.5]
$\begin{bmatrix}
 0 & \cdots & 1 \\
 2 & \cdots & 3 
\end{bmatrix}$
\end{LTXexample}

\begin{LTXexample}[pos=l, rframe={}, width=.5]
$\begin{vmatrix}
 a & b \\
 c & d 
\end{vmatrix}$
\end{LTXexample}

\begin{LTXexample}[pos=l, rframe={}, width=.5]
$\begin{Vmatrix}
 0 & 1 \\
 2 & 3 
\end{Vmatrix}$
\end{LTXexample}

\subsection{Mehrzeilige Gleichungen}
\begin{LTXexample}[pos=l, rframe={}, width=.5]
\begin{align*}
V_{Max} & = 12345\\
& = 2345\\
& = 345
\end{align*}
\end{LTXexample}

\begin{LTXexample}[pos=l, rframe={}, width=.5]
\begin{alignat*}{2}
V_{Max} & = 12345 \qquad & \text{Axiom 1}\\
& = 2345 & \text{Axiom 2}\\
& = 345  & \text{Axiom 3}
\end{alignat*}
\end{LTXexample}

\begin{LTXexample}[pos=l, rframe={}, width=.5]
$\begin{array}{lc@{\qquad}l}
1221 & test & 1221\\
23332 & testtext & 23332
\end{array}$
\end{LTXexample}


\subsection{Beispiele}
\begin{displaymath}
\begin{array}{lc@{\qquad}l}
&a \bmod b  &a\;\backslash bmod\;b\\
&x \equiv a \pmod {b}  &x\;\backslash equiv\;a\;\backslash pmod \{b\}\\
&\lim\limits_{x \to 0} \frac{\sin x}{x}  &\backslash lim _{} \{x \backslash to\;0\} \backslash frac \{\backslash sin\;x\}\{x\}\\
&\sum\limits^{\infty}_{n=1}(T \geq n) &\backslash sum\backslash limits\hat\; \{\backslash infty\}\_\{n=1\}(T \backslash geq\;n)
\end{array}
\end{displaymath}

\newpage
\section{Programmcode einbinden}
\subsection{Java Code}
\begin{LTXexample}[pos=l, rframe={}, width=.45]
\lstset{language=Java, basicstyle=\small\mdseries, tabsize=3, frame=shadowbox, framexleftmargin=8mm, xleftmargin=10mm, rulesepcolor=\color{blue}, numbers=left, numberstyle=\tiny, stepnumber=1, numbersep=12pt, backgroundcolor=\color{white}, commentstyle=\itshape\color{darkgreen}, keywordstyle=\color{darkblue}, stringstyle=\color{darkred}}
\begin{lstlisting}
public static int iterativ (int n) {
	int letzte = 1;
	//for-Schleife
	for (int i = 2; i <= n; i++) {
		int temp = i;
	}
	return temp;
}
\end{lstlisting}
\end{LTXexample}


\subsection{Pseudocode}


\newpage
\section{Formatierung von Text}
\subsection{Fett, kursiv,\dots}
\begin{LTXexample}[pos=l, rframe={}, width=.45]
\textbf{fetter Testtext}\\
\textit{kursiver Testtext}\\
\underline{unterstrichener Testtext}\\
\underline{\underline{2-fach unterstrichener Testtext}}\\
\texttt{Schreibmaschinen-Testtext}\\
\textsc{Kapit�lchen Testtext}\\
\textrm{proportionale Serifenschrift}\\
\textsf{proportionale serifenlose Schrift}
\end{LTXexample}


\subsection{Textsatz}
Standardm��ig wird Text im Blocksatz, also links- und rechtsb�ndig gesetzt. Es gibt jedoch auch f�r die anderen Satze, bestimmte Umgebungen:
\begin{multicols}{3}
\color{blue}\begin{verbatim}
\begin{center}
 Text
\end{center}
\begin{flushleft} 
 Text
\end{flushleft}
\begin{flushright}
 Text
\end{flushright}
\end{verbatim}
\end{multicols}
\color{black}

Die Befehle \textcolor{red}{\tt \bck centering \bck raggedleft \bck raggedright} verursachen keinen vertikalen Abstand!\\ z.B.: \textcolor{blue}{\tt \{\bck centering Text\}}

\subsection{Schriftgr��e}
Die Schriftgr��e �ndert man mit einem Befehl ohne Parameter. Die neue Gr��e gilt bis zur n�chsten �nderung, dem Ende der aktuellen Umgebung oder bei z.B. $\{\backslash large \dots \}$ bis zur schlie�enden Klammer.
\begin{LTXexample}[pos=l, rframe={}, width=.45]
\tiny{Testtext (tiny)}\\
\scriptsize{Testtext (scriptsize)}\\
\footnotesize{Testtext (footnotesize)}\\
\small{Testtext (small)}
\normalsize{ -- default}\\
\large{Testtext (large)}\\
\Large{Testtext (Large)}\\
\LARGE{Testtext (LARGE)}\\
\huge{Testtext (huge)}\\
\Huge{Testtext (Huge)}\\
\end{LTXexample}


\subsection{Fu�noten}
\textcolor{blue}{\tt \bck footnote\{\textit{Inhalt der Fu�note}\}} $\lrr$ erzeugt eine Fu�note im Text und gibt den Fu�noteninhalt automatisch in kleiner Schrift am Seitenende aus.\\

\mspa\note{Die Nummerierung wird in den Dokumentklassen \textsf{book} und \textsf{report} kapitelweise durchgef�hrt und in der Dokumentklasse \textsf{article} fortlaufend durch das ganze Dokument.}
\mspa\Note{Fu�noten sind nur im \textit{Absatz}-Modus m�glich, \underbar{nicht} aber im \textit{Mathematik}-Modus oder im \textit{LR}-Modus.}

Der Inhalt einer Fu�note darf beliebig komplex sein und insbesondere mehrzeiligen Text und mathematische Formeln enthalten.

\Note{Hochstellungen wie bei den Fu�notennummern lassen sich mit dem Kommando\\ \textcolor{blue}{\tt \bck textsuperscript\{Hochgestellter Text\}} erzeugen, so \textcolor{cyan}{wie\textsuperscript{hier}}.}


\ssp\subsection{Kommentare}
Kommentare werden durch \color{red}\% \color{black} eingeleitet und reichen bis zum Zeilenumbruch der aktuellen Zeile.\\
Eine M�glichkeit mehrzeilige Kommentare zu erzeugen gibt es nicht!

\subsection{Bereiche (Scope)}
Wenn gewisse Formatierungen (Schriftfarbe, Zeilenh�he, Schriftgr��e) nur in einem bestimmten Bereich wirken sollen (nicht im ganzen weiteren Text), dann kann der Bereich mit {\color{red}{geschweiften Klammern}} umschlossen werden.\\
Die Formatierungen werden dann innerhalb dieser Klammern definiert und gelten dann nur in diesem Scope.

\subsection{Mehrspaltiger Text}
\begin{LTXexample}[pos=l, rframe={}, width=.45]
\begin{multicols}{2}
Dieser ziemlich d�mmliche Text soll als Demonstration dienen, wie das Mehr-Spalten-Layout funktioniert und wann und wo die Zeilen und W�rter umgebrochen werden und wie der Textfluss ist.
\end{multicols}
\end{LTXexample}


\subsection{Besondere Zeichen in \LaTeX{}}
\begin{displaymath}
\begin{array}{lc@{\qquad}l@{\hspace{2.0cm}}c@{\qquad}l@{\hspace{2.0cm}}c@{\qquad}l}
&\_  	&\backslash\_ &\backslash  	&\backslash backslash &\textasciitilde  	&\backslash textasciitilde\\
&\S  	&\backslash S &\$  	&\backslash\$ &\&  	&\backslash \&\\
&\#  	&\backslash\# &\%  	&\backslash\% &\hat\: &\backslash hat 
\end{array}
\end{displaymath}


\subsection{Das EURO-Symbol}
{\color{blue}
\begin{verbatim}
\usepackage{eurosym}
...
\euro\end{verbatim}}



\subsection{Verlinkungen im Text}
Mit dem Paket {\tt hyperref} werden automatisch Hyperlinks vom Verweis zur Marke gesetzt.\\
Das Paket sollte am besten als letztes geladen werden, da sehr viele Einstellungen von diesem �berschrieben werden. Das Laden und das Setup der Einstellungen erfolgt z.B. mit folgendem Code:
\color{blue}\begin{verbatim}
\usepackage{hyperref}
\hypersetup{
	pdfpagemode=FullScreen,          % set default mode of PDF display
	pdfpagelayout=SinglePage,        % set layout of PDF pages
	pdfstartview={Fit|FitV|FitH},    % wie das Dokument beim �ffnen an das Fenster angepasst wird
	pdftitle={TeX Befehle},                 % Titel des PDF-Dokuments
	pdfauthor={Christian Schwabe},          % Autor(Innen) des PDF-Dokuments	
	pdfsubject={Kurzbeschr. als ein Satz},  % Inhaltsbeschreibung des PDF-Dokuments
	pdfkeywords={Stichw�rter},              % Stichwortangabe zum PDF-Dokument
	bookmarks=true,               % Lesezeichen erzeugen
	bookmarksopen=false,          % Lesezeichen ausgeklappt
	bookmarksnumbered=false,      % Anzeige der Kapitelzahlen am Anfang der Namen der Lesezeichen
	pdfstartpage=Zahl,            % Seite, welche automatisch ge�ffnet werden soll
	baseurl=http://www.abc.de/name.pdf,     % URL des PDF-Dokuments (oder Hintergrundinformationen)
	breaklinks=true,    % erm�glicht einen Umbruch von URLs
	colorlinks=true,    % Einf�rbung von Links
	linkcolor=red,      % Dokument-interne Links
	urlcolor=magenta,   % externe URL-Links
	citecolor=green,    % Links zum Literaturverzeichnis
	anchorcolor=black,  % Ankerfarbe
	filecolor=cyan,     % Links zu lokalen Dateien (file://)
	menucolor=red,      % Men�-Links
	linkbordercolor={1 0 0},    % roter Rahmen um Dokument-interne Links
	citebordercolor={0 1 0},    % gr�ner Rahmen um Links zum Literaturverzeichnis
	urlbordercolor={0 .5 .5}    % cyan Rahmen um externe URL-Links
\end{verbatim}
\color{black}

\ssp Die \textbf{bordercolor} Optionen sind nur dann sichtbar wenn die Option \textcolor{red}{\tt colorlinks=false} gesetzt wurde, d.h. wenn die Links keine Farbe haben, sondern nur ein K�stchen um sie herum stehen soll. \textbf{Dieses K�stchen wird nur angezeigt aber nicht ausgedruckt}.\\
Die \textit{Farbe} wird bei den \textit{bordercolor} Optionen als \textbf{RGB Wert zwischen 0..1} festgelegt.


\Ssp Ein internes Sprung-/ Linkziel wird definiert durch:
\color{blue}\Sspa\begin{verbatim}
	\hypertarget{name}{text}
\end{verbatim}\color{black}
Ein klickbarer Link wird definiert durch:
\color{blue}\Sspa\begin{verbatim}
	\hyperlink{name(=HypertargetName)}{text}
\end{verbatim}\color{black}
URL's werden folgenderma�en dargestellt:
\color{blue}\Sspa\begin{verbatim}
	\href{http://www.caipiranha.de}{Caipi Homepage}
\end{verbatim}\color{black}


\newpage
\section{Listen und Aufz�hlungen}
\subsection{Aufz�hlung mit Punkten}
\begin{LTXexample}[pos=l, rframe={}, width=.45]
\begin{itemize}
	\item Punkt 1
	\begin{itemize}
		\item Punkt 1
		\item Punkt 2
		\begin{itemize}
			\item Punkt 1
			\item Punkt 2
		\end{itemize}
	\end{itemize}
	\item Punkt 2
	\item blablabla
\end{itemize}
\end{LTXexample}


\subsection{Aufz�hlung mit Nummern}
\begin{LTXexample}[pos=l, rframe={}, width=.45]
\begin{enumerate}
	\item Punkt 1
	\begin{enumerate}
		\item Punkt 1
		\item Punkt 2
		\begin{enumerate}
			\item Punkt 1
			\item Punkt 2
		\end{enumerate}
	\end{enumerate}
	\item Punkt 2
	\item blablabla
\end{enumerate}
\end{LTXexample}



\subsection{Aufz�hlung mit Markierungsworten}
ist fast sowas wie ein Stickwortverzeichnis:
\begin{LTXexample}[pos=l, rframe={}, width=.45]
\begin{description}
	\item[Eselsturm,] der: an romanischen Kirchen mit gewendelter
		Rampe, auf der das Baumaterial von Eseln hinaufgetragen wurde.
	\item[Feld,] das: viereckige, polygonale oder krummlinig
		umrahmte Fl�che an W�nden, Decken, Gew�lben.
	\item[Grede,] die (lat.): Freitreppe.
\end{description}
\end{LTXexample}



\subsection{Aufz�hlung mit benutzerdefinierten Zeichen}
Diese Zeichen ben�tigen das Package "`bbding"'.\\
\begin{LTXexample}[pos=l, rframe={}, width=.45]
\begin{itemize}
	\item[\HandRight] Punkt 1
	\item[\PencilRight] Punkt 2
	\item[\XSolidBrush] Punkt 3
	\item[\DiamondSolid] Punkt 4
	\item[\OrnamentDiamondSolid] Punkt 5
	\item[\ArrowBoldRightStrobe] Punkt 6
	\item[\ArrowBoldDownRight] Punkt 7
	\item[\CircleSolid] Punkt 8
	\item[\Square] Punkt 9
	\item[\SquareSolid] Punkt 10
\end{itemize}
\end{LTXexample}



\subsection{Ver�nderung der Standard-Zeichen}
\begin{LTXexample}[pos=l, rframe={}, width=.45]
\renewcommand{\labelitemi}{$\circ$}
\renewcommand{\labelitemii}{$\bullet$}
\renewcommand{\labelitemiii}{$\diamond$}
\begin{itemize}
	\item Punkt xx
	\begin{itemize}
		\item Punkt yy
		\item Punkt yyy
		\begin{itemize}
			\item Punkt 1
			\item Punkt 2
		\end{itemize}
	\end{itemize}
	\item Punkt 2
	\item blablabla
\end{itemize}
\end{LTXexample}

\begin{LTXexample}[pos=l, rframe={}, width=.45]
\renewcommand{\labelenumi}{\alph{enumi})}
\renewcommand{\labelenumii}{\alph{enumi}.\Roman{enumii}}
\renewcommand{\labelenumiii}{\arabic{enumiii}.)}
\begin{enumerate}
	\item Punkt 1
	\begin{enumerate}
		\item Punkt 1
		\item Punkt 2
		\begin{enumerate}
			\item Punkt 1
			\item Punkt 2
		\end{enumerate}
	\end{enumerate}
	\item Punkt 2
	\item blablabla
\end{enumerate}
\end{LTXexample}


\newpage
\section{Eigene Kommandos definieren}
\subsection{Allgemein}
Syntax zum definieren von eigenen Kommandos:\\
\textit{$\backslash$newcommand$\{\backslash$neuerName$\}\{$Definition$\}$}\\

Man kann den neuen Kommandos auch Argumente �bergeben (maximal 9 Argumente!):\\
\textit{$\backslash$newcommand$\{\backslash$neuerName$\}$[Anzahl der Argumente]$\{$Definition$\}$}\\


\subsection{Eigene Kommandos}
\begin{LTXexample}[pos=l, rframe={}, width=.45]
\FALSCH \\
\RICHTIG \\
\Aufg{3} \\
\Loes{4} \\
\UAufg{6} \\
\mathee{x^2}
\Mathee{x^2}
x\Igl y \\
x\igl y \\
$\rar$ \\
$\lrr$ \\
vertikale Abst�nde: \\
\ ssp = 2mm \\
\ Ssp = 3mm \\
\ msp = 5mm \\
\ Msp = 8mm \\
\einruck{test}
\txt{f�r Text in Math-Umgebungen} \\
\tdgrey{farbiger Text} \\
\tblue{auch in red, orange und dgreen} \\
$\vek{0 & 1\\ 2 & 3}$ \\
\cen{zentrierter Text}
\end{LTXexample}


\newpage
\section{Tabellen und Tabulatoren}
\hypertarget{tabul}{\subsection{Tabellen}}
Tabellen werden mit der \texttt{tabular}-Umgebung definiert.
\color{blue}
\begin{verbatim}
\begin{tabular}[<Position>]{<Spaltendefinition>}
...
<Eintrag>   &  <Eintrag>   \\
...
\end{tabular}
\end{verbatim}
\color{black}
\storestyleof{itemize}
\begin{listliketab}
\begin{tabularx}{\textwidth}{lX}
	\textbullet & Jede Zeile wird mit dem Zeilenumbruch $\backslash\backslash$ abgeschlossen.\\
	\textbullet & Spalten werden mit Hilfe des \&-Zeichens voneinander getrennt.\\
	\textbullet & Bei \texttt{<Spaltendefinition>} gibt man die Anzahl der Spalten in der Form \texttt{|l r c|} an. �ber die Zeichen \textbf{r}ight \textbf{l}eft und \textbf{c}enter wird die Textausrichtung in der Spalte definiert.\\
	\textbullet & Weitere m�gliche Formateintr�ge:\\
	&{\begin{tabularx}{\linewidth}{@{}llX@{}}
		\tiny{\SquareSolid} & \textcolor{blue}{\tt{p\{breite\}}} & Absatz der Breite 										\textit{breite}\\
		\tiny{\SquareSolid} & \textcolor{blue}{\tt{*\{anzahl\}\{format\}}} & das 												\textit{format} wird \textit{anzahl} mal wiederhohlt, z.B. \textit{*\{3\}\{c|\}} 							$\hat=$ \textit{\{c|c|c|\}}\\
		\tiny{\SquareSolid} & \textcolor{blue}{\texttt{@\{text\}}} & der Text \texttt{text} 						wird anstatt des normal verwendeten Zwischenraumes zwischen den benachbarten Spalten 					angebracht.\\
		\end{tabularx}
	}\\
	\textbullet & \textbf{Die Anzahl der Ausrichtungszeichen legt die Spaltenzahl fest!}\\
	\textbullet & Mithilfe des Befehls \textcolor{blue}{\tt \textbackslash multicolumn\{<Spaltenzahl>\}\{Ausrichtung\}\{Text\}} kann ein Eintrag �ber mehrere Spalten gehen.\\
	\textbullet & \textcolor{blue}{\tt $\backslash$hline} f�r eine horizontale Linie �ber die ganze Tabellenbreite (anzugeben direkt nach einem \textbackslash\textbackslash).\\
	\textbullet & \textcolor{blue}{\tt $\backslash$cline\{von--bis\}} f�r eine horizontale Linie �ber einzelne Spalten (wie {\tt $\backslash$hline} anzugeben direkt nach einem \textbackslash\textbackslash).\\
	\textbullet & \textcolor{blue}{\tt $\backslash$vline} f�r eine vertikale Linie innerhalb einer Spalte.\\
\end{tabularx}
\end{listliketab}

\begin{LTXexample}[pos=l, rframe={}, width=.50]
\begin{tabular}{l|c|r||p{2.5cm}|r@{.}l}
links & zentriert & rechts & Dies ist ein Text, der wie ein Absatz der Breite 2,5 cm formatiert wird. & DM 2 & 50 \\ 
\hline
\multicolumn{2}{c|}{eins und zwei} & drei 1 \vline{} 2 & Absatz & \$ 20 & 50 \\ 
\cline{1-2}
\end{tabular}
\end{LTXexample}

\subsubsection{Zeilenabstand}
Falls $\backslash$hline verwendet wird, ist die Spaltenh�he etwas zu klein. Dazu gibt es 3 M�glichkeiten dies zu korrigieren:

\paragraph{Arraystretch}
$\;$ \\
Kann auch verwendet werden, um generell die Zeilenh�he zu verringern oder erh�hen.\\
{\color{orange}{F�gt Platz �ber und unter dem Zeilentext ein!}}
{\color{blue}
\begin{verbatim}
\renewcommand{\arraystretch}{1.2}\end{verbatim}}

\paragraph{Extrarowheight}
$\;$ \\
{\color{orange}{F�gt nur Platz �ber dem Zeilentext ein! F�gt aber auch Platz ein, falls gar kein $\backslash$hline in der Zeile vorhanden ist!}}
{\color{blue}
\begin{verbatim}
\usepackage{array}
...
\setlength{\extrarowheight}{1.5pt}\end{verbatim}}

\paragraph{Bigstruts}
$\;$ \\


\hypertarget{tabul}{\subsubsection{Spaltenabstand}}



\subsection{Tabulatoren}
Hier muss man selbst f�r die Breite der Spalten sorgen!\\
\storestyleof{itemize}
\begin{listliketab}
\begin{tabularx}{\linewidth}{llX}
	\textbullet \, & \textcolor{blue}{\tt $\backslash$=} &  setzt einen Tabulator an der aktuellen Spaltenposition.\\
	\textbullet & \textcolor{blue}{\tt $\backslash$>} & springt um eine Tabulatorposition nach rechts. \\
	\textbullet & \textcolor{blue}{\tt $\backslash$<} & springt um eine Tabulatorposition nach links.\\
	\textbullet & \textcolor{blue}{\tt $\backslash$+} & verschiebt den linken Rand um eine Tabulatorposition nach rechts (mu� vor dem Zeilenumbruch mit \verb+\\+ erfolgen!).\\
	\textbullet & \textcolor{blue}{\tt $\backslash$-} & verschiebt den linken Rand um eine Tabulatorposition nach links (mu� vor dem Zeilenumbruch mit \verb+\\+ erfolgen!).\\
	\textbullet & \textcolor{blue}{\tt $\backslash$pushtabs} & speichert die aktuellen Tabulatorpositionen auf dem Stapel und l�scht dann die aktuellen (verschachtelbar).\\
	\textbullet & \textcolor{blue}{\tt $\backslash$poptabs} & l�scht die aktuellen Tabulatorpositionen und l�dt die gespeicherten vom Stapel an deren Stelle (verschachtelbar).\\
	\textbullet & \textcolor{blue}{\tt $\backslash$kill} & entfernt die aktuelle Zeile (meist eine Musterzeile).\\
	\textbullet & \textcolor{blue}{\tt $\backslash\backslash$} & beendet die aktuelle Zeile.\\
\end{tabularx}
\end{listliketab}
\begin{LTXexample}[pos=l, rframe={}, width=.45]
\begin{tabbing}
erste Spalte breit \= zweite Spalte \=
dritte Spalte \kill
erste Spalte \> zweite Spalte \> dritte Spalte \\
vorne \> mitte \> hinten \+ \\
mitte \> hinten \+ \\
hinten \- \- \\
vorne \> mitte \> hinten \\
diese Zeile erscheint nicht \kill
\end{tabbing}
\end{LTXexample}

\vspace{1.0cm}
\subsection{listliketab-Package}
Erm�glicht es Listen mit Tabellen zu kombinieren und so Tabulatoren innerhalb von Listen zu verwenden.\\
Anstelle der \hyperlink{tabul}{{\tt tabular}}--Umgebung innerhalb von {\tt listliketab}, k�nnen auch \hyperlink{tabulx}{{\tt tabularx}}, \hyperlink{longt}{{\tt longtable}} und evtl andere verwendet werden.\\
\textcolor{blue}{\tt \textbackslash usepackage\{listliketab\}}
\begin{LTXexample}[pos=l, rframe={}, width=.45]
\storestyleof{itemize}
\begin{listliketab}
\begin{tabular}{lll}
	\textbullet & OneOneOne & TWOtwo\\
	\textbullet & Two & Three\\
	\textbullet & Three & FourfourFour \\
\end{tabular}
\end{listliketab}
\end{LTXexample}

\begin{LTXexample}[pos=l, rframe={}, width=.35]
\storestyleof{enumerate}
\begin{listliketab}
\newcounter{tabenum}\setcounter{tabenum}{0}
\newcommand{\nextnum}{\addtocounter{tabenum}{1}\alph{tabenum})}
\begin{tabular}{l>{\bf}l@{~or~}>{\bf}l@{~or~}>{\bf}l}
	\nextnum & Red & green & blue\\
	\nextnum & Short & stout & tall\\
	\nextnum & Happy & sad & confused\\
\end{tabular}
\end{listliketab}
\end{LTXexample}

\hypertarget{tabulx}{\subsection{tabularx-Package}}
\color{blue}
\begin{verbatim}
\usepackage{tabularx}
  \begin{tabularx}{Breite}{Spaltendefinition}
 	  K�rper
  \end{tabularx}\end{verbatim}\color{black}
Diese Umgebung ist nah verwandt mit der \hyperlink{tabul}{{\tt tabular}}-Umgebung, somit kann man sich auch an deren Syntax orientieren. Die Zellen werden durch \& getrennt und mit \textbackslash\textbackslash   beendet.\\
Hier kann man allerdings die Tabellenbreite explizit vorgeben. Die mit X markierten Spalten sind Spalten variabler Breite. Zwischen diesen Spalten wird der verbliebene Platz aufgeteilt, der sich aus der Gesamtbreite und der schon verbrauchten Breite f�r die "`normalen"'Spalten ergibt.\\
\storestyleof{itemize}
\begin{listliketab}
\begin{tabularx}{\textwidth}{llX}
	\textbullet \, & {\it Breite} & Hier wird die gew�nschte Tabellenbreite angegeben. H�ufig verwendet werden hier \textcolor{blue}{\tt\textbackslash textwidth} oder \textcolor{blue}{\tt\textbackslash linewidth}.\\
	\textbullet & {\it X} & neue Spaltendefinition $\rightarrow$ Spalte mit variabler Breite.\\
	\textbullet & \textcolor{blue}{\tt \textbackslash hsize} & sind 2 oder mehr Spalten mit X markiert w�rde der verbliebene Platz gleichm��ig auf diese aufgeteilt werden. Mit {\tt \textbackslash hsize} l�sst sich ein Verh�ltniszwischen diesen variablen Spalten angeben.\\
	\textbullet & \multicolumn{2}{X}{Will man mehrere tabularx Umgebungen schachteln, muss man diese in geschweifte Klammern einschlie�en.}
\end{tabularx}
\end{listliketab}


\hypertarget{longt}{\subsection{longtable-Package}}
\color{blue}
\begin{verbatim}
\usepackage{longtable}
  \begin{longtable}{ Format }
    K�rper 
  \end{longtable}\end{verbatim}
\color{black}
Wie die \hyperlink{tabul}{{\tt tabular}} Umgebung, erlaubt jedoch auch mehrseitige Tabellen.\vspace{0.2cm}\\
Um die Tabelle automatisch auf die Seiten umzubrechen, muss LaTeX die Tabelle mit seinem Seitenumbruchsalgorithmus verareiten k�nnen. Dazu ist es jedoch notwendig, das Dokument mehrfach zu kompilieren.\\
Um diese Anzahl der Kompiliervorg�nge zu verringern, kann die erste Zeile der Tabelle mit {\tt \textbackslash kill} anstatt mit {\tt \textbackslash\textbackslash} abgeschlossen werden.\\

\subsection{ltxtable-Package}
\color{blue}
\begin{verbatim}
\usepackage{ltxtable}
  \LTXtable{Breite}{Dateiname}
\end{verbatim}
\color{black}
Tabelle als \hyperlink{longt}{{\tt longtable}}--Umgebung in eine eigene Datei schreiben und dabei die X-Spezifikation von \hyperlink{tabulx}{{\tt tabularx}} benutzen.\\
Danach Tabelle mit obiger Anweisung in das Dokument einf�gen.\\

Eine andere M�glichkeit w�re mit dem \hyperlink{fileco}{{\tt filecontents}}--Package zu arbeiten. Hier kann die Tabellendefinition im gleichen Dokument erfolgen. Dabei wird die Tabelle dann in die entsprechende Datei geschrieben (�berschrieben) und an der gew�nschten Stelle eingef�gt.
\color{blue}
\begin{verbatim}
\documentclass[a4paper,12pt]{scrreprt}

\usepackage{filecontents}
\begin{filecontents*}{LONGTAB.tex}
\begin{longtable}{lX}
EINS &
Auf der Mauer
\\
DREI &
sitzt 'ne kleine Wanze
\end{longtable}
\end{filecontents*}


\usepackage[german]{babel}
\usepackage[latin1]{inputenc}

\usepackage{ltxtable}

\begin{document}
\LTXtable{\textwidth}{LONGTAB}
\end{document}
\end{verbatim}
\color{black}



\subsection{\textsf{caption} bei longtable und ltxtable}
Normalerweise ist es etwas schwer bis unm�glich diese beiden Arten in eine Gleitumgebung einzubetten um so eine Tabellenbeschreibung drunter zu setzen. Dies l�sst sich aber auch ohne eine Gleitumgebung, durch dieses Konstrukt realisieren:
\einruck{ \textcolor{blue}{\tt \textbf{\bck begin}\{longtable\}\{Format\}} \\ \textcolor{blue}{\tt \bck caption\{mein-Text\}\bck label\{tab:LV1\}\textbf{\bck endlastfoot}}}





\subsection{Einf�rben von Tabellen(-zellen)}
\einruck{\textcolor{blue}{\tt \bck usepackage\{colortbl\}}}
\begin{LTXexample}[pos=l, rframe={}, width=.4]
\storestyleof{enumerate}
\renewcommand{\arraystretch}{1.2}
\begin{tabularx}{.95\textwidth}{|XX|}
\hline
\rowcolor[gray]{.6}
{\tt book}, {\tt report} & {\tt article}\\
\hline
\textbf{\tt \bck chapter} &  \\
\rowcolor{lightgray}
\textbf{\tt \bck section} & \textbf{\tt \bck section}\\
\textbf{\tt \bck subsection} & \textbf{\tt \bck subsection}\\
\rowcolor{lightgray}
\textbf{\tt \bck subsubsection} & \textbf{\tt \bck subsubsection}\\
\textbf{\tt \bck paragraph} & \textbf{\tt \bck paragraph}\\
\rowcolor{lightgray}
\textbf{\tt \bck subparagraph} & \textbf{\tt \bck subparagraph}\\
\hline
\end{tabularx}
\end{LTXexample}



\subsection{Gleitobjekte}
\textit{Tabellen} und \textit{Abbildungen} sind oft gr��ere Boxen, die sich nur m�hsam so in einem Text anordnen lassen, dass keine h��lichen Leerr�ume auf Seiten entstehen. In den meisten Publikationen sind Tabellen und Bilder daher nicht unbedingt genau an dem Platz, an dem sie angesprochen werden, sondern befinden sich h�ufig zu Beginn oder am Ende einer Seite. Typischerweise gibt es durchnummerierte Bild- bzw. Tabellenbeschreibungen, die dann zitiert werden k�nnen.
\begin{table}[!htb]
\centering
\renewcommand{\arraystretch}{1.2}
\begin{tabularx}{.75\textwidth}{|cX|}
\hline
\rowcolor[gray]{.6}
Option & Bedeutung\\
\hline
{\tt h} & Versuche, das Objekt an genau dieser Stelle zu platzieren\\
\rowcolor{lightgray}
{\tt t} & Versuche, das Objekt am Seiten\textbf{kopf} zu platzieren\\
{\tt b} & Versuche, das Objekt am Seiten\textbf{ende} zu platzieren\\
\rowcolor{lightgray}
{\tt p} & Versuche, das Objekt auf einen eigenen Gleitobjektseite zu platzieren\\
{\tt !} & Die Versuche sollen ohne gro�e R�cksichtnahme auf \LaTeX-Steuerungsgr��en durchgef�hrt werden (erh�ht die Wahrscheinlichkeit, dass das Gleitobjekt an der vermuteten Stelle gesetzt wird)\\
\hline
\end{tabularx}
\caption[Platzierungsoptionen von Gleitobjekten]{Optionen zum Platzieren von Gleitobjekten}\label{tab:tab3}
\end{table}

\LaTeX\, unterst�tzt die Erstellung solcher Gleitobjekte, die nicht notwendig an der Stelle erscheinen, an der sie im Quelltext stehen. Statt dessen bem�ht sich \LaTeX, diese Objekte an \grqq g�nstige\grqq\, Stellen gleiten zu lassen.\\

\sspa Gleitende Tabellen werden erzeugt durch die Umgebung:\\
\textcolor{blue}{\tt \textbf{\bck begin}\{table\}[\textit{Optionen}] \textit{Gleitender Tabelleninhalt} \textbf{\bck end}\{table\}}\\

\sspa Gleitende Abbildungen werden erzeugt durch die Umgebung:\\
\textcolor{blue}{\tt \textbf{\bck begin}\{figure\}[\textit{Optionen}] \textit{Gleitender Abbildungsinhalt} \textbf{\bck end}\{figure\}}\\

Durch \grqq \textcolor{red}{!}\grqq\, kann man \LaTeX\, anweisen, die Platzierung m�glichst rasch unter Verletzung interner Vorgaben vorzunehmen, z.B.
\einruck{\textcolor{cyan}{\tt \textbf{\bck begin}\{table\}[!htb]}}
Die Reihenfolge ist nicht entscheidend, sondern nur \textit{ob} eine der Optionen gesetzt ist!\\
Wenn ein Gleitobjekt nicht auf der aktuellen Seite platziert werden konnte, so wird es in eine interne Liste gesteckt. Auf jeder neuen Seite versucht \LaTeX, die Objekte aus der Liste zu verwenden. Blockiert ein Objekt diese Liste, so kann sich ein Stau aus Gleitobjekten bilden. Ein solcher Stau l�sst sich durch einen \grqq gewaltsamen\grqq\, Seitenumbruch mit \textcolor{red}{\tt \bck clearpage} bzw. \textcolor{red}{\tt \bck cleardoublepage} bereinigen, da hierbei alle Gleitobjekte der Liste ausgegeben werden.


\subsubsection{Beschriftung von Gleitobjekten}
Beschriftungstexte werden �ber den Befehl
\einruck{\textcolor{blue}{\tt \textbf{\bck caption}[\textit{Kurztext}]\{\textit{Beschreibungstext}\}}}
erstellt. Der optionale \textit{Kurztext} dient bei langen Beschreibungen als Abk�rzung in Inhaltsverzeichnissen. Diese Beschriftung wird automatisch nummeriert und kann mit einem \textcolor{blue}{\tt \textbf{\bck label}}-Befehl versehen werden.
\einruck{\textcolor{cyan}{\tt \textbf{\bck caption}\{Beispieltabelle\}\textbf{\bck label}\{tab:gleittabelle\}}}



\subsubsection{Anpassung der Beschriftungen}
Statt normale Textschrift zu verwenden, kann man z.B. eine kleinere und serifenfreie verwenden, um diese Beschriftung vom Resttext abzuheben. Diese und andere �nderungen lassen sich mit dem \textcolor{red}{Paket \texttt{\textsf{caption}}} vornehmen.
\einruck{\textcolor{blue}{\tt \textbf{\bck usepackage}\{caption\}}}
Eine gutes Beispiel:
\einruck{\textcolor{cyan}{\tt \textbf{\bck usepackage}[format=hang, font=\{footnotesize,sf\}, labelfont=\{bf\}, margin=1cm, aboveskip=5pt, position=bottom]\{caption\}}}



\newpage
\section{Farben}

\subsection{Standard-Farben}
\begin{table}[!htb]
\centering
\begin{tabular}{*{5}{rp{1.5cm}}}
\colbox{black} & black  & \colbox{darkgray} & darkgray &  \colbox{lime} & lime & \colbox{pink} & pink & \colbox{violet} & violet\\
\colbox{blue} & blue & \colbox{gray} & gray & \colbox{magenta} & magenta & \colbox{purple} & purple & \colbox{white} & white \\ 
\colbox{brown} & brown & \colbox{green} & green & \colbox{olive} & olive & \colbox{red} & red & \colbox{yellow} & yellow \\ 
\colbox{cyan} & cyan & \colbox{lightgray} & lightgray & \colbox{orange} & orange & \colbox{teal} & teal &  &  \\ 
\end{tabular} 
\caption{Standardfarben}\label{tab:tab5}
\end{table}
\Sspa Diese Farben sind immer verf�gbar!



\subsection{Zusatzfarben}
Diese Farben sind �ber die \textsf{\textit{svgnames}} Option des \textsf{\textit{xcolor}} Paketes zus�tzlich verf�gbar:
\einruck{\textcolor{blue}{\tt \bck usepackage[svgnames]\{xcolor\}}}
\ssp
\LTXtable{\textwidth}{LongTabFarb.tex}




\subsection{Schrift und Hintergrund}
\begin{itemize}
	\item \texttt{$\backslash$ color\{red\}:} \color{red} Der folgende Text ist rot bis zum n�chsten Farbwechsel.\color{black}
	\item \texttt{$\backslash$ textcolor\{green\}}\{\textcolor{green}{Der eingeklammerte Text ist gr�n}\}.
	\item \texttt{$\backslash$ pagecolor\{blue\}:}\;Setzen der Seitenhintergrundfarbe.
\end{itemize}

\subsection{Farbboxen}
\begin{itemize}
	\item \texttt{$\backslash$ colorbox\{red\}}\{\colorbox{red}{Rot hinterlegte Box}\}.
	\item \texttt{$\backslash$ fcolorbox\{blue\}\{green\}}\{\fcolorbox{blue}{green}{Gr�ne Box mit blauem Rand}\}.
	\item �ndern der Randst�rke mit \texttt{$\backslash$setlength\{$\backslash$fboxrule\}\{5pt\}}: {\setlength{\fboxrule}{5pt}\fcolorbox{blue}{green}{5pt Rand}}
	\item �ndern des Randabstandes mit \texttt{$\backslash$setlength\{$\backslash$fboxsep\}\{0pt\}}: {\setlength{\fboxsep}{0pt}\fcolorbox{blue}{green}{5pt Rand}}
\end{itemize}

\subsection{Das \texttt{framed}-Paket}
\begin{itemize}
	\item \textcolor{blue}{\tt \textbackslash usepackage\{framed\}}
	\item Zeilen- und Seitenumbr�che innerhalt des Rahmens m�glich!
	\item rahmt per Voreinstellung �ber die gesamte Seitenbreite ein!
	\item die definiteren Umgebungen sind:
	\begin{itemize}
		\item framed
		\item shaded
		\item snugshade
		\item leftbar
	\end{itemize}
\end{itemize}

\begin{LTXexample}[pos=l, rframe={}, width=.40]
\imp{TEST}
\end{LTXexample}

\begin{LTXexample}[pos=l, rframe={}, width=.40]
\defi{TEST}
\end{LTXexample}

\begin{LTXexample}[pos=l, rframe={}, width=.40]
\bsp{\centerline{TEST}}
\end{LTXexample}

\begin{LTXexample}[pos=l, rframe={}, width=.40]
\note{TEST}
\end{LTXexample}

\begin{LTXexample}[pos=l, rframe={}, width=.40]
\Note{TEST}
\end{LTXexample}

\begin{LTXexample}[pos=l, rframe={}, width=.40]
\code{Code Test}
\end{LTXexample}

\begin{LTXexample}[pos=l, rframe={}, width=.40]
\colorfbox{0.6}{orange}{wichtig}{und text}
\end{LTXexample}

\begin{LTXexample}[pos=l, rframe={}, width=.40]
\colorfbox{0.95}{blue}{\textcolor{white}{test2}}{\centerline{und text-zwei}}
\end{LTXexample}


\section{Aufbau und Organisation}

\subsection{Dokumentklassen}
\begin{table}[!htb]
\centering
\renewcommand{\arraystretch}{1.2}
\begin{tabularx}{.75\textwidth}{|lX|}
\hline
\rowcolor[gray]{.6}
Klassenname & Zielsetzung\\
\hline
{\tt book} & Komplette B�cher mit Kapitelgliederung\\
\rowcolor{lightgray}
{\tt report} & L�ngere Berichte mit Kapitelgliederung, z.B. technischer Bericht oder Diplomarbeiten\\
\rowcolor{white}
{\tt article} & Artikel in wissenschaftlichen Zeitschriften oder k�rzere Berichte mit Abschnittsgliederung\\
\rowcolor{lightgray}
{\tt letter} & Briefe\\
\rowcolor{white}
{\tt scrbook} & Europ�sisierte Variante von {\tt book} (KOMA-Script Paket)\\
\rowcolor{lightgray}
{\tt scrreprt} & Europ�sisierte Variante von {\tt report} (KOMA-Script Paket)\\
\rowcolor{white}
{\tt scrartcl} & Europ�sisierte Variante von {\tt article} (KOMA-Script Paket)\\
\hline
\end{tabularx}
\caption[Dokumentklassen der Pr�ambel]{Auswahl an \LaTeX -Dokumentklassen}\label{tab:tab1}
\end{table}
Daneben gibt es noch zahlreiche Varianten zur Erzeugung von Overheadfolien und Beamerpr�sentationen.




\subsection{Logische Gliederung}
\begin{table}[!htb]
\centering
\renewcommand{\arraystretch}{1.2}
\begin{tabularx}{.5\textwidth}{|XX|}
\hline
\rowcolor[gray]{.6}
{\tt book}, {\tt report} & {\tt article}\\
\hline
\textbf{\tt \bck chapter} &  \\
\rowcolor{lightgray}
\textbf{\tt \bck section} & \textbf{\tt \bck section}\\
\textbf{\tt \bck subsection} & \textbf{\tt \bck subsection}\\
\rowcolor{lightgray}
\textbf{\tt \bck subsubsection} & \textbf{\tt \bck subsubsection}\\
\textbf{\tt \bck paragraph} & \textbf{\tt \bck paragraph}\\
\rowcolor{lightgray}
\textbf{\tt \bck subparagraph} & \textbf{\tt \bck subparagraph}\\
\hline
\end{tabularx}
\caption[Gliederungsbefehle zur logischen Strukturierung]{Gliederungsbefehle in hierarchischer Anordnung}\label{tab:tab2}
\end{table}

\begin{description}
\item[\textcolor{blue}{\tt \bck section[\textit{Kurztitel}]\{Titel\}}] $\lrr$ verwendet \textit{Kurztitel} im Inhaltsverzeichnis
\item[\textcolor{blue}{\tt \bck section*\{Titel\}}] $\lrr$ erzeugt eine unnummerierte �berschrift ohne Eintrag in das Inhaltsverzeichnis
\item[\textcolor{blue}{\tt \bck setcounter\{secnumdepth\}\{4\}}] $\lrr$ eine Erh�hung dieser Zahl bewirkt die Nummerierung der n�chsttieferen Gliederungsebenen, da die Nummerierung gew�hnlich nur bis zur Ebene \textbf{\tt \bck subsubsection} ($\igl$3) statt findet.
\item[\textcolor{blue}{\tt \bck part\{\textit{Name des Dokumentteils}\}}] $\lrr$ zerteilt ein Dokument in (gro�e) Teile ohne Auswirkung auf die Nummerierung.
\end{description}


\subsubsection{weitere Gliederungsbefehle f�r B�cher}
\begin{description}
\item[\textcolor{blue}{\tt \bck frontmatter}] $\lrr$ Buchanfang: Seiten erhalten r�mische Ziffern und Abschnitte haben keine Nummern
\item[\textcolor{blue}{\tt \bck mainmatter}] $\lrr$ wird vor dem ersten Kapitel verwendet: Seitennummerierung
wird zur�ckgesetzt und mit arabischen Ziffern ausgegeben
\item[\textcolor{blue}{\tt \bck appendix}] $\lrr$ Beginn des Anhangs: Nummerierung der Kapitel
wird dann auf fortlaufende Buchstaben umge�ndert
\item[\textcolor{blue}{\tt \bck backmatter}] $\lrr$ das Ende des eigentlichen Buches, worauf �blicherweise
das Literaturverzeichnis und das Stichwortverzeichnis folgen
\end{description}



\subsection{Querverweise, Label}
\begin{description}
\item[\textcolor{blue}{\tt \bck label\{\textit{Markierung}\}}] $\lrr$ erzeugt eine unsichtbare \textit{Markierung} (Label) im Dokument
\item[\textcolor{blue}{\tt \bck ref\{\textit{Markierung}\}}] $\lrr$ nimmt Bezug auf das referenzierte Element, gibt die \textbf{Nummer} von \textit{Markierung} aus!
\item[\textcolor{blue}{\tt \bck pageref\{\textit{Markierung}\}}] $\lrr$ gibt die \textbf{Seitenzahl} aus, auf der die \textit{Markierung} liegt
\end{description}

\Ssp Markierbar und damit zitierbar sind nur bestimmte Elemente:
\begin{itemize}
\item Gliederungsbefehle ({\tt \bck section\{Titel\}}, {\tt \bck chapter\{Titel\}}, ...)
\item Tabellen und Abbildungen in Gleitumgebungen ({\tt \bck begin\{table\}}, {\tt \bck begin\{figure\}})
\item mathematische Formeln
\item mathematische Definitionen, S�tze, Theoreme usw.
\item Elemente die nicht markierbar w�ren kann man durch eine Phantomsection ebenfalls markieren:\\
		\textcolor{blue}{\tt \bck phantomsection}\\
		\textcolor{blue}{\tt \bck label\{\textit{key1}\}}
\end{itemize}

\Ssp \textit{Markierung} dient dabei als \textbf{eindeutiger Bezeichner} und sollte dem Strukturelement entsprechend genannt werden, f�r das sie verwendet wird:
\begin{itemize}
\item \textcolor{blue}{\tt \bck chapter\{\textcolor{red}{Foo}\}\bck label\{\textcolor{red}{cha:foo}\}}
\item \textcolor{blue}{\tt \bck section\{\textcolor{red}{Baa}\}\bck label\{\textcolor{red}{sec:baa}\}}
\item \textcolor{blue}{\tt \bck subsection\{\textcolor{red}{Buzz}\}\bck label\{\textcolor{red}{ssec:buzz}\}}
\item \textcolor{blue}{\tt \bck begin\{\textcolor{red}{table}\}\bck label\{\textcolor{red}{tab:boom}\}\bck end\{table\}}
\end{itemize}




\subsection{Indexregister}
\begin{itemize}
	\item \textcolor{blue}{\tt \textbackslash usepackage\{makeidx\}}
	\item \textcolor{blue}{\tt \textbackslash makeindex}
	\item beides direkt untereinander und noch vor dem Beginn des Dokuments!
	\item Einf�gen von \textcolor{blue}{\tt \textbackslash printindex} vor dem Dokument-Ende (bzw. an der Stelle, wo das Indexregister erscheinen soll)
	\item An den Stellen/W�rtern, die in den Index eingetragen werden sollen, folgendes anf�gen:\\
			\textcolor{blue}{\tt \textbackslash index\{Indexeintrag\_bzw\_Wort\}}
	\item oder \textcolor{blue}{\tt \textbackslash index\{Eintrag!Untereintrag\}}
	\item \textcolor{orange}{\tt \textbackslash index\{Virtuell@Eintrag\}}\\
			Virtuelle Eintr�ge sind notwendig, um Sonderzeichen oder mathematische Symbole in den Index einzuordnen\\
			\textcolor{orange}{\tt \textbackslash index\{wunschenswert@w�nschenswert\}}\\
			\textcolor{orange}{\tt \textbackslash index\{R@\textbackslash R\}}
	\item um statt \textit{Index} ein anderes Wort zu verwenden:\\
			\textcolor{blue}{\tt \bck renewcommand\{\bck indexname\}\{anderes\_Wort\}}
	\item um das \textit{Indexregister} in das \textit{Inhaltsverzeichnis} mit aufzunehmen:\\
			\textcolor{blue}{\tt \bck addcontentsline\{toc\}\{section\}\{\bck indexname\}}
	\item diese beiden Befehle am besten dann direkt vor \textcolor{blue}{\tt \bck printindex} am Ende des Dokuments
	\item es muss ein extra Compilier-Durchlauf mit \textcolor{red}{\tt makeindex \%.idx} stattfinden!!
\end{itemize}

\subsubsection{Index mit richtigen Seitenzahlen}
Um zu verhindern das im Inhaltsverzeichnis f�r den Index falsche Seitenzahlen auftauchen m�ssen die folgenden beiden Befehle vor \textcolor{blue}{\tt \bck printindex} am Ende des Dokumentes eingef�gt werden:
\einruck{\textcolor{blue}{\tt \bck clearpage\\\bck phantomsection}}


\subsubsection{Index mit richtigem Seitenlayout}
Um auch beim Index das gleiche Seitenlayout wie bei allen anderen (vorherigen) Seiten zu behalten muss bei der Definition der Kopf- und Fu�zeilen, nach \textcolor{blue}{\tt \bck pagestyle\{scrheadings\}} noch folgende Zeile eingef�gt werden:
\einruck{\textcolor{blue}{\tt \bck renewcommand*\{\bck indexpagestyle\}\{scrheadings\}}}


\subsection{Glossar}
\subsubsection{Pr�ambel bzw. Header}
\begin{itemize}
	\item \textcolor{blue}{\tt \bck usepackage[nonumberlist,toc,section,numberline,translate=false]\{glossaries\}}
	\item \textcolor{red}{\textbf{Das Paket muss unbedingt erst NACH den folgenden Paketen geladen werden!!}}
	\begin{itemize}
		\item html
		\item inputenc
		\item babel
		\item ngerman
		\item \textcolor{red}{\textbf{hyperref !!!}}
	\end{itemize}
	\item Die Bedeutung der einzelnen Paket-Optionen:
		\begin{itemize}
			\item \textcolor{blue}{\tt nonumberlist} $\igl$ um zu verhindern, dass im Glossar nach den einzelnen Beschreibungen noch Seitenzahlen angezeigt werden
			\item \textcolor{blue}{\tt toc} $\igl$ Glossar im Inhaltsverzeichnis anzeigen lassen
			\item \textcolor{blue}{\tt section} $\igl$ im Inhaltsverzeichnis auf der Section-Ebene erscheinen lassen
			\item \textcolor{blue}{\tt numberline} $\igl$ richtet den Eintrag im Inhaltsverzeichnis nicht an den Section-Nummern sondern an den Section-Namen aus! (also etwas einger�ckt)
			\item \textcolor{blue}{\tt translate=false} $\igl$ f�r die eigene Umbenennung von \grqq Glossary\grqq\; (siehe gleich hier drunter)
		\end{itemize}
	\item direkt darunter um \grqq Glossary\grqq\; letztendlich in (z.B.) \grqq Glossar\grqq\; umzubenennen:\\ 
			\textcolor{blue}{\tt \bck renewcommand*\{\bck glossaryname\}\{Glossar\}}
	\item direkt danach:\\ 
			\textcolor{blue}{\tt \bck makeglossaries}
	\item wie beim Indexregister am besten alles untereinander, vor Beginn des Dokuments aber nach dem Indexregister!!
	\item Einf�gen von \textcolor{blue}{\tt \bck printglossaries} vor dem Dokument-Ende (bzw. an der Stelle, wo das Glossar erscheinen soll)
	\item um zu Verhindern, das am Ende jeder Beschreibung ein Punkt erscheint:\\
			\textcolor{blue}{\tt \bck renewcommand*\{\bck glspostdescription\}\{\}}
\end{itemize}

\subsubsection{Verwendung}
\begin{itemize}
	\item um einen neuen Glossar-Eintrag zu erstellen (am besten in der Pr�ambel - hinter \textit{\tt usepackage}):\\
			\textcolor{blue}{\tt \bck newglossaryentry\{abk\}\{name=Abk�rzung, description=\{blahbla\}\}}
	\item Verwendung der Glossar-Eintr�ge im (Haupt-)Text:
		\begin{itemize}
			\item \textcolor{blue}{\tt \bck glspl\{abk\}}\\
					f�gt \textit{Abk�rzung\textbf{s}} mit Verlinkung zum Eintrag im Glossar ein (falls das \textit{hyperref Package} verwendet wird)
			\item \textcolor{blue}{\tt \bck gls\{abk\}}\\
					f�gt \textit{Abk�rzung} \textbf{ohne} Plural aber auch mit Verlinkung ein
			\item \textcolor{blue}{\tt \bck Gls\{abk\}}\\
								f�gt \textit{Abk�rzung} mit beginnendem Gro�buchstaben ein (f�r einen Satzanfang)
			\item \textcolor{blue}{\tt \bck Glspl\{abk\}}\\
								f�gt \textit{Abk�rzung\textbf{s}} mit beginnendem Gro�buchstaben ein
			\item \textcolor{blue}{\tt \bck GLS\{abk\}}\\
								f�gt \textit{ABK�RZUNG} komplett in Kapit�lchen ein
			\item \textcolor{blue}{\tt \bck glsentrytext\{abk\}}\\
								\textcolor{red}{f�r die Verwendung in besonderen Bereichen wie: \bck chapter, \bck section, \bck caption} - wird nicht verlinkt!!
		\end{itemize}
\end{itemize}


\subsubsection{Vorraussetzungen und Compilierung}
\begin{enumerate}
	\item \textbf{Perl} muss installiert sein!! z.B. \href{http://strawberryperl.com/}{\textcolor{blue}{Strawberry Perl}}
	\item Die Datei \textbf{makeglossaries.bat} (\textcolor{red}{\tt C:\bck Program Files\bck MiKTeX\bck scripts\bck glossaries}) muss ge�ndert werden:\\
			statt \grqq {\tt perl}\grqq\; muss der Pfad zur \underbar{Perl.exe} dort eingetragen werden, z.B. \textcolor{red}{\tt \grqq C:\bck strawberry\bck perl\bck bin\bck perl.exe\grqq}
	\item Bei der \textbf{Compilierung} der PDF muss ein weiterer Durchlauf (nach {\tt makeindex}) hinzugef�gt werden:\\
			\textcolor{red}{\tt "C:/Program Files/MiKTeX/scripts/glossaries/makeglossaries.bat"\;\%}
\end{enumerate}




\subsection{Abk�rzungsverzeichnis}
\begin{itemize}
	\item \textcolor{blue}{\tt \bck usepackage[\textbf{\underbar{acronym}}]\{glossaries\}} muss in der Optionen-Liste enthalten sein!
	\item Definition von Abk�rzungen: \textcolor{blue}{\tt \bck newacronym\{PD\}\{PDF\}\{Portable Dokument Format\}}
	\item Verwendung: \textcolor{blue}{\tt \bck gls\{PD\}} $\rar$ im Text erscheint \grqq PDF\grqq\; mit Verlinkung auf das Abk�rzungsverzeichnis
	\item das Abk�rzungsverzeichnis wird automatisch mit dem Glossar durch Verwendung von \textcolor{blue}{\tt makeglossaries} erstellt
	\item deutsche Umbenennung: \textcolor{blue}{\tt \bck renewcommand*\{\bck acronymname\}\{Abk�rzungsverzeichnis\}}
\end{itemize}



\subsection{Symbolverzeichnis}
\begin{itemize}
	\item \textcolor{blue}{\tt \bck newglossary[slg]\{symblist\}\{syi\}\{syg\}\{Symbolverzeichnis\}} muss noch vor \textcolor{blue}{\tt makeglossaries} stehen!!
	\item Definition: \textcolor{blue}{\tt \bck newglossaryentry\{symb:Pi\}\{name=\$\bck pi\$, description=\{Die Kreiszahl.\}, sort=symbolpi, type=symblist\}}
	\item Verwendung: \textcolor{blue}{\tt \bck gls\{symb:Pi\}}
	\item Die Eintr�ge der Symbolliste sollten immer einen \textit{sort}-Eintrag bekommen, wobei auf den gleichen Anfangsbuchstaben zu achten ist, um die Leerr�ume zwischen den gruppierten Eintr�gen zu vermeiden.
\end{itemize}


\subsection{Literaturverzeichnis}
\begin{itemize}
	\item eine \textcolor{red}{\tt .bib}-Datei muss erzeugt werden / worden sein (z.B. durch Citavi-Export)
	\item in der Pr�ambel / Header einzuf�gen:\\
			\textcolor{blue}{\tt \bck usepackage[square]\{natbib\}}\\
			\textcolor{blue}{\tt \bck bibliographystyle\{natdin\}}
	\item in der BA/SA wird statt dessen folgendes verwendet:\\
				\textcolor{blue}{\tt \bck usepackage\{bibgerm\}}\\
				\textcolor{blue}{\tt \bck bibliographystyle\{geralpha\}}\\
				\textcolor{red}{es wird dann auch nur {\tt \bck cite} ohne \grqq p\grqq\, verwendet!!}
	\item vor \textcolor{blue}{\tt \bck end\{document\}} einf�gen:\\
			\textcolor{blue}{\tt \bck addcontentsline\{toc\}\{section\}\{Literatur\}}\\
			... um das Literaturverzeichnis	im Inhaltsverzeichnis aufzuf�hren\\
			\textcolor{blue}{\tt \bck bibliography\{DATEINAME\}}\\
						... DATEINAME der \textcolor{red}{\tt .bib}-Datei um das Literaturverzeichnis einzuf�gen
	\item Verwendung: generell ist der Teil aus der \textcolor{red}{\tt .bib}-Datei zu verwenden, der nach der ersten \{-Klammer folgt: \textcolor{red}{\tt @article\{Koomey2011b,}\; $\rar$ also \textcolor{red}{\tt Koomey2011b}
	\begin{itemize}
		\item \textcolor{blue}{\tt \bck citep\{Koomey2011b\}} $\lrr$ \textbf{\tt [Koomey u.a., 2011]}
		\item \textcolor{blue}{\tt \bck citep\{Koomey2011b\}} $\lrr$ \textbf{\tt [Koomey, 2011]}
		\item \textcolor{blue}{\tt \bck citep[chap.\~\,2]\{Koomey2011b\}} $\lrr$ \textbf{\tt [Koomey, 2011, chap. 2]}
		\item \textcolor{blue}{\tt \bck citep[siehe][]\{Koomey2011b\}} $\lrr$ \textbf{\tt [siehe Koomey, 2011]}
		\item \textcolor{blue}{\tt \bck citep\{Koomey2011, Koomey2011b\}} $\lrr$ \textbf{\tt [Koomey, 2011; Koomey u.a., 2011]}
		\item \textcolor{blue}{\tt \bck nocite\{Koomey2011\}} $\lrr$ f�gt den Eintrag zum Literaturverzeichnis hinzu ohne eine Referenz im Text daf�r zu erstellen!
	\end{itemize}
\end{itemize}



\subsubsection{Literaturtypen und zugeh�rige Felder}
F�r die unterschiedlichen Literaturtypen sind jeweils verschiedene Felder ben�tigt bzw. optional. Hier ein �berblick:
\renewcommand{\arraystretch}{1.2}
\LTXtable{\textwidth}{LongTab.tex}




\subsection{Abbildungs- und Tabellenverzeichnis}
\begin{itemize}
	\item um eines von beiden oder auch beide Verzeichnisse in das Inhaltsverzeichnis mit aufzunehmen, muss in die Optionen von \textcolor{blue}{\tt \bck documentclass[...]\{...\}} folgendes eingef�gt werden:\\
			\textcolor{red}{\tt toc=listof}	
	\item An der Stelle im Dokument wo eines der beiden Verzeichnisse entstehen soll, muss eingef�gt werden:\\
			\textcolor{blue}{\tt \bck listoftables}
			\textcolor{blue}{\tt \bck listoffigures}
	\item Verwendung: in \textcolor{red}{\tt table} oder \textcolor{red}{\tt figure}-Gleitumgebungen durch den \textcolor{blue}{\tt \bck caption} Befehl:\\
			\textcolor{blue}{\tt \bck  caption[Kurzbeschr.-die-im-VZ-erscheint]\{Tabellen-oder-Abbildungs-Unterschrift\}\bck label\{tab:tble2 | fig:figu2\}}
\end{itemize}

\end{document}