\newpage
\section{Farben}

\subsection{Standard-Farben}
\begin{table}[!htb]
\centering
\begin{tabular}{*{5}{rp{1.5cm}}}
\colbox{black} & black  & \colbox{darkgray} & darkgray &  \colbox{lime} & lime & \colbox{pink} & pink & \colbox{violet} & violet\\
\colbox{blue} & blue & \colbox{gray} & gray & \colbox{magenta} & magenta & \colbox{purple} & purple & \colbox{white} & white \\ 
\colbox{brown} & brown & \colbox{green} & green & \colbox{olive} & olive & \colbox{red} & red & \colbox{yellow} & yellow \\ 
\colbox{cyan} & cyan & \colbox{lightgray} & lightgray & \colbox{orange} & orange & \colbox{teal} & teal &  &  \\ 
\end{tabular} 
\caption{Standardfarben}\label{tab:tab5}
\end{table}
\Sspa Diese Farben sind immer verf�gbar!



\subsection{Zusatzfarben}
Diese Farben sind �ber die \textsf{\textit{svgnames}} Option des \textsf{\textit{xcolor}} Paketes zus�tzlich verf�gbar:
\einruck{\textcolor{blue}{\tt \bck usepackage[svgnames]\{xcolor\}}}
\ssp
\LTXtable{\textwidth}{LongTabFarb.tex}




\subsection{Schrift und Hintergrund}
\begin{itemize}
	\item \texttt{$\backslash$ color\{red\}:} \color{red} Der folgende Text ist rot bis zum n�chsten Farbwechsel.\color{black}
	\item \texttt{$\backslash$ textcolor\{green\}}\{\textcolor{green}{Der eingeklammerte Text ist gr�n}\}.
	\item \texttt{$\backslash$ pagecolor\{blue\}:}\;Setzen der Seitenhintergrundfarbe.
\end{itemize}

\subsection{Farbboxen}
\begin{itemize}
	\item \texttt{$\backslash$ colorbox\{red\}}\{\colorbox{red}{Rot hinterlegte Box}\}.
	\item \texttt{$\backslash$ fcolorbox\{blue\}\{green\}}\{\fcolorbox{blue}{green}{Gr�ne Box mit blauem Rand}\}.
	\item �ndern der Randst�rke mit \texttt{$\backslash$setlength\{$\backslash$fboxrule\}\{5pt\}}: {\setlength{\fboxrule}{5pt}\fcolorbox{blue}{green}{5pt Rand}}
	\item �ndern des Randabstandes mit \texttt{$\backslash$setlength\{$\backslash$fboxsep\}\{0pt\}}: {\setlength{\fboxsep}{0pt}\fcolorbox{blue}{green}{5pt Rand}}
\end{itemize}

\subsection{Das \texttt{framed}-Paket}
\begin{itemize}
	\item \textcolor{blue}{\tt \textbackslash usepackage\{framed\}}
	\item Zeilen- und Seitenumbr�che innerhalt des Rahmens m�glich!
	\item rahmt per Voreinstellung �ber die gesamte Seitenbreite ein!
	\item die definiteren Umgebungen sind:
	\begin{itemize}
		\item framed
		\item shaded
		\item snugshade
		\item leftbar
	\end{itemize}
\end{itemize}

\begin{LTXexample}[pos=l, rframe={}, width=.40]
\imp{TEST}
\end{LTXexample}

\begin{LTXexample}[pos=l, rframe={}, width=.40]
\defi{TEST}
\end{LTXexample}

\begin{LTXexample}[pos=l, rframe={}, width=.40]
\bsp{\centerline{TEST}}
\end{LTXexample}

\begin{LTXexample}[pos=l, rframe={}, width=.40]
\note{TEST}
\end{LTXexample}

\begin{LTXexample}[pos=l, rframe={}, width=.40]
\Note{TEST}
\end{LTXexample}

\begin{LTXexample}[pos=l, rframe={}, width=.40]
\code{Code Test}
\end{LTXexample}

\begin{LTXexample}[pos=l, rframe={}, width=.40]
\colorfbox{0.6}{orange}{wichtig}{und text}
\end{LTXexample}

\begin{LTXexample}[pos=l, rframe={}, width=.40]
\colorfbox{0.95}{blue}{\textcolor{white}{test2}}{\centerline{und text-zwei}}
\end{LTXexample}
