% Eine �berschrift erster Ordnung machen
\section{Installation}
\subsection{zus�tzliches Profil: LaTeX => PS => PDF}
Wird ben�tigt zum Einbinden von Bilder und vor allem wenn mit dem Paket PSTricks gearbeitet wird!
\begin{itemize}
	\item Am besten erst \href{http://www.ghostscript.com/}{\textcolor{blue}{Ghostscript}} installieren - noch bevor TeXnicCenter installiert wird!! Dann wird das Profil automatisch erstellt!
	\item Wurde das vers�umt, dann kann man \href{http://www.ghostscript.com/}{\textcolor{blue}{Ghostscript}} auch im nachhinein installieren, muss dann aber folgende Schritte ausf�hren:	
	\begin{enumerate}
		\item Das Profil \textbf{LaTeX => PS} duplizieren
		\item unter \textbf{Nachbearbeitung} $\lrr$ \textbf{Postprozessoren}		
		\begin{itemize}
			\item \textbf{DviPs} \\
					$\lrr$ Anwendung: \textcolor{red}{\texttt{C:\textbackslash Program Files\textbackslash MiKTeX 2.8\textbackslash miktex\textbackslash bin\textbackslash dvips.exe}}\\
					$\lrr$ Argumente: \textcolor{red}{\texttt{-t a4 "'\%Bm.dvi"'}}
			\item \textbf{Ghostscript (ps2pdf)} hinzuf�gen\\
					$\lrr$ Anwendung: \textcolor{red}{\texttt{C:\textbackslash Program Files\textbackslash MiKTeX 2.8\textbackslash miktex\textbackslash bin\textbackslash ps2pdf.exe}}\\
					$\lrr$ Argumente: \textcolor{red}{\texttt{"'\%bm.ps"'}}
		\end{itemize}
		\item bei \textbf{Viewer} dann das Gleiche wie beim Profil \textbf{LaTeX => PDF} verwenden.
	\end{enumerate}
\end{itemize}


\Ssp\subsection{MikTeX + TeXnicCenter}
In den Umgebungsvariablen (System -> Erweiterte Systemeinstellungen -> Erweitert -> Umgebungsvariablen -> Path)\\
den Pfad zum MikTeX-bin-Ordner hinzuf�gen: \textcolor{red}{\texttt{C:$\backslash$Program Files$\backslash$MikTeX$\backslash$miktex$\backslash$bin$\backslash$;}}

\Ssp\subsection{Verwendung von TikZ und TeXnicCenter}
\begin{enumerate}
	\item \textcolor{red}{gnuplot} downloaden und ins C-Programmverzeichnis entpacken: \href{http://www.gnuplot.info/}{\textcolor{blue}{http://www.gnuplot.info/}} 
	\item In den Umgebungsvariablen (System -> Erweiterte Systemeinstellungen -> Erweitert -> Umgebungsvariablen -> Path)\\
den Pfad zum gnuplot-binary-Ordner hinzuf�gen: \textcolor{red}{\texttt{C:$\backslash$Program Files$\backslash$gnuplot$\backslash$binary$\backslash$}}
	\item Im Programm TeXnicCenter eine Kommandozeilen Option hinzuf�gen:\\
			\texttt{-interaction=nonstopmode} $\enspace\longrightarrow\enspace$ \texttt{--src -interaction=nonstopmode --enable-write18 "`\%Wm"'}\\
			unter Ausgabe -> Augabeprofile definieren -> Argumente die an den Compiler �bergeben werden sollen
	\item neben \textcolor{red}{\texttt{\textbackslash usepackage\{tikz\}}} sollte, um m�glichst viele/alle Funktionen nutzen zu k�nnen, auch noch \\ \textcolor{red}{\texttt{\textbackslash usetikzlibrary\{automata,positioning,decorations,shadows,fadings,arrows,snakes,\\ backgrounds,petri,shapes.geometric\}}} eingebunden werden.\\
\end{enumerate}


\vspace*{-1em}\subsubsection{E/R-Diagramme mit TikZ}
\begin{itemize}
	\item hierzu wird zus�tzlich \textcolor{red}{\texttt{\textbackslash usepackage\{tikz-er2\}}} ben�tigt
	\item ich habe eine eigene etwas angepasste Datei die per \textcolor{red}{\texttt{\textbackslash usepackage\{C://Users/Cheffe/Documents/tikz-er2\}}} eingebunden werden muss!
\end{itemize}


\Ssp\subsection{Verwendung von PDF-XChange Viewer und TeXnicCenter}
Bei Verwendung des PDF-XChange Viewer als Standard-PDF-Viewer in TeXnicCenter, k�nnen die PDF-Dokumente vor dem Kompilieren automatisch geschlossen werden:
\begin{enumerate}
	\item In TexnicCenter auf \textbf{Ausgabe} klicken
	\item \textbf{Ausgabeprofil definieren} 
	\item \textbf{Viewer}
	\item Beim \textbf{Pfad der Anwendung} den Pfad zum Viewer angeben, z.B: \textcolor{red}{\texttt{D:$\backslash$Programme$\backslash$Tracker Software$\backslash$PDF-XChange Viewer$\backslash$pdf-viewer$\backslash$PDFXCview.exe}}
	\item \textbf{Projektausgabe betrachten} $\longrightarrow$ Kommandozeile ausw�hlen und als Kommando \textcolor{red}{\texttt{"'\%bm.pdf"'}} eintragen 
	\item Bei \textbf{Compilierung vor Ausgabe schlie�en} auch Kommandozeile ausw�hlen und \textcolor{red}{\texttt{/close "'\%bm.pdf"'}} eintragen\\
\end{enumerate}


\subsection{Verwendung von SumatraPDF und TeXnicCenter (incl. SyncTeX)}
Sumatra PDF ist ein extrem schlanker PDF-Viewer, der �nderungen an ge�ffneten PDF akzeptiert und sogar Vorw�rts- und R�ckw�rtssuche untest�tzt, d.h. das Springen zwischen entsprechenden Stellen in der PDF-Ausgabe und dem Quellcode.\\
Nach dem Installieren von Sumatra PDF mit den Standardeinstellungen wird das Profil LaTeX $\Rightarrow$ PDF mit dem neuen Namen \textcolor{red}{LaTeX $\Rightarrow$ PDF (Sumatra)} kopiert. Folgende Einstellungen werden ge�ndert:
\begin{itemize}
	\item \textbf{(La)Tex}\\Bei \textbf{Argumente, die an den Compiler �bergeben werden sollen} muss folgendes eingetragen werden:\\\textcolor{red}{\texttt{-synctex=-1 -interaction=nonstopmode "'\%pm"'}}
	\item \textbf{Viewer}\\Bei \textbf{Pfad der Anwendung} muss folgendes stehen:\\
		\textcolor{red}{\texttt{C:$\backslash$Programme$\backslash$SumatraPDF$\backslash$SumatraPDF.exe -reuse-instance -inverse-search}}\\		\textcolor{red}{\texttt{"$\backslash$"C:$\backslash$Programme$\backslash$TeXnicCenter$\backslash$TEXCNTR.EXE$\backslash$" /ddecmd $\backslash$"[goto('\%f', '\%l')]$\backslash$"'"'}}\\Hinweis: nach dem \texttt{search} kommt ein Leerzeichen!
	\item Bei \textbf{Projektausgabe betrachten} muss der Radiobutton bei \textcolor{red}{Kommandozeile} gesetzt werden.\\
			Au�erdem ist das Kommando \textcolor{red}{\texttt{[Open("'\%bm.pdf"',0,0,1)]}} einzugeben.
	\item Bei \textbf{Suche in Ausgabe} muss der Radiobutton bei \textcolor{red}{DDE-Kommando} gesetzt werden mit:\\ \textbf{Server:} 	\textcolor{red}{\texttt{SUMATRA}}\\\textbf{Thema:} \textcolor{red}{\texttt{Control}}\\
			Au�erdem ist das Kommando \textcolor{red}{\texttt{[ForwardSearch("'\%bm.pdf"',"'\%Wc"',\%l,0,0,0)]}} einzugeben.
	\item Bei \textbf{Vor Kompilierung Ausgabe schlie�en} muss der Radiobutton bei \textcolor{red}{Nicht schlie�en} gesetzt werden.\\
\end{itemize}


\Ssp\subsection{Versionierung mit GoogleCode und SVN}
\begin{itemize}
	\item neues Google Projekt erstellen auf der Webseite \href{http://code.google.com/hosting/}{\textcolor{blue}{http://code.google.com/hosting/}} �ber den Link unten:\\ \textbf{Create a new project}
	\item \href{http://tortoisesvn.tigris.org/}{\textcolor{blue}{TortoiseSVN}} in Windows installieren 
	\item zum Ein- und Auschecken wird noch das spezielle googlecode.com Passwort ben�tigt.\\
			Dies ist zu finden unter: \href{https://code.google.com/hosting/settings}{\textcolor{blue}{https://code.google.com/hosting/settings}} 
	\item im \textbf{Windows-Explorer $\lrr$ TortoiseSVN $\lrr$ Import}\\(um den ausgew�hlten Ordner/Datei zum Projekt hinzuzuf�gen)\\
	Dabei drauf achten dass der Ordner \textbf{\texttt{trunk}} verwendet wird!
	\item \textbf{Username}: \texttt{abcdz0815@googlemail.com}
	\item \textbf{Passwort}: siehe oben das spezielle GoogleCode-Passwort
	\item \textbf{Pfad}: \textcolor{red}{\texttt{https://PROJEKTNAME.googlecode.com/svn/trunk}}
	\item dann \textbf{SVN Auschecken} um in dem gew�hlten Ordner eine lokale Projekt-Version zu erzeugen
	\item jetzt bestimmte Dateien (die zu oft bei der Bearbeitung von Tex-Dateien ge�ndert werden) noch vom Ein-/Auschecken ausschlie�en:
			\begin{itemize}
				\item zuerst eine Datei \textit{svnignore} anlegen mit dem Inhalt:
				\begin{verbatim}
				*.dvi
				*.ps
				*.pdf
				*.rtf
				*.bbl
				*.blg
				*.aux
				*.idx
				*.ilg
				*.ind
				*.log
				*.toc
				*.tps				
				\end{verbatim}
				\item dann per Kommandozeile in dem Ordner ausf�hren: \textcolor{red}{\texttt{svn propset svn:ignore -F svnignore .}}
				\item Kommandozeile mit \textcolor{red}{\texttt{exit}} wieder beenden
			\end{itemize}
	\item Datei \textit{svnignore} wieder l�schen und das Ganze per \textbf{SVN �bertragen}, updaten
	\item jetzt kann normal gearbeitet werden, lediglich beim \textbf{L�schen} muss dies �ber das TortoiseSVN-Men� geschehen (Dateien/Ordner werden dann auch erst beim n�chsten SVN Update wirklich gel�scht!)
\end{itemize}