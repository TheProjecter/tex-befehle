\newpage
\section{Formatierung von Text}
\subsection{Fett, kursiv,\dots}
\begin{LTXexample}[pos=l, rframe={}, width=.45]
\textbf{fetter Testtext}\\
\textit{kursiver Testtext}\\
\underline{unterstrichener Testtext}\\
\underline{\underline{2-fach unterstrichener Testtext}}\\
\texttt{Schreibmaschinen-Testtext}\\
\textsc{Kapit�lchen Testtext}\\
\end{LTXexample}


\subsection{Textsatz}
Standardm��ig wird Text im Blocksatz, also links- und rechtsb�ndig gesetzt. Es gibt jedoch auch f�r die anderen Satze, bestimmte Umgebungen:
\begin{multicols}{3}
\color{blue}\begin{verbatim}
\begin{center}
 Text
\end{center}
\begin{flushleft} 
 Text
\end{flushleft}
\begin{flushright}
 Text
\end{flushright}
\end{verbatim}
\end{multicols}
\color{black}

\subsection{Schriftgr��e}
Die Schriftgr��e �ndert man mit einem Befehl ohne Parameter. Die neue Gr��e gilt bis zur n�chsten �nderung, dem Ende der aktuellen Umgebung oder bei z.B. $\{\backslash large \dots \}$ bis zur schlie�enden Klammer.
\begin{LTXexample}[pos=l, rframe={}, width=.45]
\tiny{Testtext (tiny)}\\
\scriptsize{Testtext (scriptsize)}\\
\footnotesize{Testtext (footnotesize)}\\
\small{Testtext (small)}\\
\normalsize{default}\\
\large{Testtext (large)}\\
\Large{Testtext (Large)}\\
\LARGE{Testtext (LARGE)}\\
\huge{Testtext (huge)}\\
\Huge{Testtext (Huge)}\\
\end{LTXexample}


\subsection{Kommentare}
Kommentare werden durch \color{red}\% \color{black} eingeleitet und reichen bis zum Zeilenumbruch der aktuellen Zeile.\\
Eine M�glichkeit mehrzeilige Kommentare zu erzeugen gibt es nicht!

\subsection{Bereiche (Scope)}
Wenn gewisse Formatierungen (Schriftfarbe, Zeilenh�he, Schriftgr��e) nur in einem bestimmten Bereich wirken sollen (nicht im ganzen weiteren Text), dann kann der Bereich mit {\color{red}{geschweiften Klammern}} umschlossen werden.\\
Die Formatierungen werden dann innerhalb dieser Klammern definiert und gelten dann nur in diesem Scope.

\subsection{Mehrspaltiger Text}
\begin{LTXexample}[pos=l, rframe={}, width=.45]
\begin{multicols}{2}
Dieser ziemlich d�mmliche Text soll als Demonstration dienen, wie das Mehr-Spalten-Layout funktioniert und wann und wo die Zeilen und W�rter umgebrochen werden und wie der Textfluss ist.
\end{multicols}
\end{LTXexample}


\subsection{Besondere Zeichen in \LaTeX{}}
\begin{displaymath}
\begin{array}{lc@{\qquad}l@{\hspace{2.0cm}}c@{\qquad}l@{\hspace{2.0cm}}c@{\qquad}l}
&\_  	&\backslash\_ &\backslash  	&\backslash backslash &\textasciitilde  	&\backslash textasciitilde\\
&\S  	&\backslash S &\$  	&\backslash\$ &\&  	&\backslash \&\\
&\#  	&\backslash\# &\%  	&\backslash\% &\hat\: &\backslash hat 
\end{array}
\end{displaymath}


\subsection{Das EURO-Symbol}
{\color{blue}
\begin{verbatim}
\usepackage{eurosym}
...
\euro\end{verbatim}}



\subsection{Verlinkungen im Text}
Mit dem Paket {\tt hyperref} werden automatisch Hyperlinks vom Verweis zur Marke gesetzt.\\
Das Paket sollte am besten als letztes geladen werden, da sehr viele Einstellungen von diesem �berschrieben werden. Das Laden und das Setup der Einstellungen erfolgt z.B. mit folgendem Code:
\color{blue}\begin{verbatim}
\usepackage{hyperref}
\hypersetup{
	pdfpagemode=FullScreen,
	pdftitle={TeX Befehle},					% Titel des PDF-Dokuments
	pdfauthor={Christian Schwabe},	% Autor(Innen) des PDF-Dokuments
	pdfpagelayout=SinglePage,
	bookmarks=true,   				% Lesezeichen erzeugen
	bookmarksopen=true,   		% Lesezeichen ausgeklappt
	bookmarksnumbered=true,   % Anzeige der Kapitelzahlen am Anfang der Namen der Lesezeichen
	pdfstartpage=Zahl,   			% Seite, welche automatisch ge�ffnet werden soll
	baseurl=http://www.server.de/dateiname.pdf,   % URL des PDF-Dokuments (oder 																																	Hintergrundinformationen)
	pdfsubject={Kurzbeschreibung als ein Satz},   % Inhaltsbeschreibung des PDF-Dokuments
	pdfkeywords={Stichw�rter},   									% Stichwortangabe zum PDF-Dokument
	breaklinks=true,   % erm�glicht einen Umbruch von URLs
	colorlinks=true,   % Einf�rbung von Links
	linkcolor=black,   % Linkfarbe: schwarz
	anchorcolor=black, % Ankerfarbe: schwarz
	citecolor=black,   % Literaturlinks: schwarz
	filecolor=black,   % Links zu lokalen Dateien: schwarz
	menucolor=black,   % Acrobat Men� Eintr�ge: schwarz
	pagecolor=black,   % Links zu anderen Seiten im Text: schwarz
	urlcolor=black		 % URL-Farbe: schwarz
\end{verbatim}
\color{black}
Ein internes Sprung-/ Linkziel wird definiert durch:
\color{blue}\begin{verbatim}
	\hypertarget{name}{text}
\end{verbatim}\color{black}
Ein klickbarer Link wird definiert durch:
\color{blue}\begin{verbatim}
	\hyperlink{name(=HypertargetName)}{text}
\end{verbatim}\color{black}
URL's werden folgenderma�en dargestellt:
\color{blue}\begin{verbatim}
	\href{http://www.caipiranha.de}{Caipi Homepage}
\end{verbatim}\color{black}
